%*************************************************
\documentclass[a4paper,10pt]{article}
\usepackage[brazil]{babel}
\usepackage[utf8]{inputenc}
%codigo
%\usepackage[latin1]{inputenc}
\usepackage{amsthm,amsfonts,amsmath,amssymb}
%img
\usepackage{graphicx}
\usepackage{subfig}
%tabela

%fim
%fim
\usepackage{listings} 
\usepackage{makeidx}
\usepackage{enumerate}
\usepackage{hyperref}
\hypersetup{
  colorlinks,
  linkcolor=blue,
  filecolor=blue,
  urlcolor=blue,
  citecolor=blue 
}

%titulo
\title{Servidor iterativo UDP}
\author{Leandro Kümmel Tria Mendes RA033910 \\ Fernando Teixeira RA085858}
\makeindex
%inicio
\begin{document}
\maketitle
\begin{figure}[!htb]
  \centering
  \includegraphics[scale=0.5]{logo.jpg}
\end{figure}
\newpage
\tableofcontents
\listoffigures
\listoftables
\newpage
\section{Introdução}
O objetivo desse projeto é implementar um sistema cliente/servidor
iterativo, com operações para o gerenciamento de livros em uma livraria.
Na comunicação entre cliente e servidor utilizou-se o protocolo UDP\footnote{
Mais informações sobre UDP \url{http://tools.ietf.org/html/rfc768} }, da camada de transporte, e a partir da execução de testes podemos avaliar alguns aspectos
desse protocolo e posteriormente compará-lo com o projeto anterior, no qual foi utilizado o TCP como protocolo.
\section{Desenvolvimento}
Linux foi o sistema operacional utilizado para o desenvolvimento (distribuição
2.6.43.8-1.fc15.i686). Igualmente, para os testes utilizou-se duas máquinas com linux, porém com distribuições diferentes.
\subsection{Protocolo UDP - User Datagram Protocol}
Diferentemente do protocolo TCP, o UDP é simplificado, ou seja, possui características diferentes do primeiro protocolo, logo foi escrito de modo a não garantir que os dados enviados (pelo
servido) e recebidos (pelo cliente) de forma correta, na sequência 
adequada, pela rede.
O protocolo UDP, User Datagram Protocol, não envia dados em \"stream\" sim em \"datagrams\footnote{Fonte: \url{http://www.freesoft.org/CIE/Course/Section3/5.htm} }\". No primeiro caso, a comunicação têm dois pontos, o dado é colocado em um final e vêem do outro ponto, não há dados duplicados, descartados, ou reorganizados. Já o \"datagrams\" são bem menores do que o primeiro, há uma fonte e destino para envio dos dados, mas não pode ser chamada de conexão, um \"datagram\" não possui relacionamento com nenhum outro, sendo assim não há garantia de entrega do pacote, porém arregura-se que os pacotes não serão enviados fragmentados. 
\\As características fundamentais do UDP são:
\begin{itemize}
\item \emph{Orientado à transação}: não há necessidade de uma conexão.
\item \emph{Simples}: utiliza-se de datagramas, sem confiabilidade de entrega.
\item \emph{Stateless}: não mantém o estado sobre uma conexão, visto que não existe uma.
\end{itemize}
\subsection{Implementação}
O projeto conta com cinco diretórios, cada um com seu Makefile (exceto relatorio e estat), arquivos
principal (main.c e main.h) e um README.md\footnote{Leia esse arquivo antes de
executar o sistema} para instruções adicionais. Há também um Makefile, o qual 
compila e executa o sistema.
\\Os diretório são:
\begin{enumerate}[I.]
\label{sec:dirs}
\item \label{itm:common} \emph{common}: Contém arquivos de uso comum, tanto pelo servidor quanto 
pelo cliente, inclusive o arquivo que calcula a média dos testes executados.
\item \label{itm:server} \emph{server}: Contém os arquivos que preparam uma porta para esperar
conexões e manipulam o sistema de livraria.
\item \label{itm:client} \emph{client}: Funções que provêm comunicação com um servidor, envio das opções
escolhidas pelo cliente e interface para as respostas do sistema de livraria
(servidor).
\item \label{itm:estat} \emph{estat}: Medidas de tempo efetuadas pelo teste.
\item \label{itm:relatorio} \emph{relatorio}
\end{enumerate}
\subsubsection{Manipulação de dados}
Todas as estruturas utilizadas para leitura/escrita dos livros são dinâmicas. \\Arquivos presentes no diretório \emph{common}[\ref{itm:common}]:
\begin{enumerate}[I.]
\item \emph{error.c error.h}: Gerencia erros que eventualmente podem ocorrer.
\item \emph{common.c common.h}: Funções de uso comum.
\item \label{itm:avl} \emph{avl.c avl.h}: Gerencia a estrutura básica da livraria, utiliza-se 
árvore AVL\footnote{ \url{http://pages.cs.wisc.edu/~ealexand/cs367/NOTES/AVL-Trees/index.html} }, pois a busca, inserção/atualização têm complexidade \boxed{O(logN)} , sendo N
o número de elementos na árvore, no caso a quantidade de livros diferentes.
\item \emph{archives.c archives.h}: Manipula arquivos. Faz a leitura do arquivo
da livraria\footnote{Ver README.md para mais detalhes do arquivo da livraria}.
\item \label{itm:tcp} \emph{tcp.c tcp.h}: Contém apenas algumas constantes, utilizadas tanto pelo protocolo TCP quanto pelo UDP.
\item \emph{books.c books.h}: Gerencia a estrutura básica de um livro e seus autores.
\item \label{itm:tempo}\emph{tempo.c tempo.h}: Gerencia a estrutura de testes, lê e escreve em arquivos localizados no diretório \emph{estat}[\ref{itm:estat}].
\item \emph{livros/livros}: Arquivo contendo os livros\footnote{Ver README.md para mais detalhes do arquivo da livraria}.
\end{enumerate}
\subsubsection{Conexão Servidor/Cliente}
Compreende dois diretórios \emph{server}[\ref{itm:server}] e \emph{client}[\ref{itm:client}]
\\\textbf{Servidor}:
\begin{enumerate}[I.]
\item \emph{server.c server.h}: Apenas inicia o servidor dada um número de uma porta.
\item \label{itm:tcpserver} \emph{udp\_server.c udp\_server.h}: Gerencia a comunicação com o cliente, em outras palavras, o udp\_server.c recebe um datagram do udp\_client.c[\ref{itm:tcpclient}] e envia uma resposta adequada ao mesmo. Como mencionado anteriormente, o UDP não garante a entrega de um datagrama, sendo assim, espera 
de um pacote, que é uma chamada do sistema (syscall)\footnote{Mais em: \url{http://www.tldp.org/LDP/khg/HyperNews/get/syscall/syscall86.html} }, é bloqueante. 
Portanto, garante-se o envio/recebimento de pelo menos 1 byte para o cliente/servidor, previnindo um bloqueio por ambas as partes envolvidas.
\item \emph{login.c login.h}: Gerencia o login necessário para editar a quantidade de um livro.
\end{enumerate}
\textbf{Cliente}:
\begin{enumerate}[I.]
\item \emph{client.c client.h}: Apenas inicia a comunicação com um servidor dado o endereço IP e um número de uma porta.
\item \label{itm:tcpclient} \emph{udp\_client.c udp\_client.h}: Gerencia leitura, da entrada dada pelo usuário do sistema, e a comunicação entre hospedeiro e cliente. Também, garante-se o envio de pelo menos 1 byte para o servidor [\ref{itm:tcpserver}].
\end{enumerate}
\subsection{Coleta e gerência de dados para testes}
Para realizar os testes implementou-se alguns arquivos adicionas[\ref{itm:tempo}]. Uma constante, denominada NUM\_TESTES\footnote{Nesse sistema consideramos NUM\_TESTES igual a 100} , contém o número de testes a serem realizados, ou seja, cada opção do \emph{menu}[\ref{itm:tcpserver}] é executada NUM\_TESTE vezes. Todos os dados são 
salvos no diretório \emph{estat}[\ref{itm:estat}].
\subsection{Vantagens da implementação}
O sistema de livraria é um sistema robusto e com baixa complexidade de tempo.
A escolha da estrutura de árvore avl[\ref{itm:avl}], possibilitou em boa 
performance em questão de tempo de processamento, uma vez que, essa estrutura
mostrou-se eficaz para o problema e tem melhor complexidade de tempo com 
relação a outras estruturas, além de ser da implementação e manuntenção serem
simples.
\\ Com relação a comunicação entre cliente/servidor, vale ressaltar que há uma troca de mensagens\footnote{Sabe-se que o UDP envia/recebe datagrams} inicial entre os dois, na qual o cliente inicia, enviando um pacote apenas para conhecimento do servidor de sua existência, já o servidor responde com um pacote cujo o conteúdo é o número de bytes da maior mensagem possível a ser enviada pelo servidor, seguido de ACK (reconhecimento), enviado pelo cliente.
Como a função \boxed{recvfrom(int sockfd, void *buf, size\_t len, int flags,struct sockaddr *src\_addr, socklen\_t *addrlen);}\footnote{\url{http://linux.die.net/man/2/recvfrom}} começa a ler o buffer, o qual é escrito o datagram enviado pelo servidor, antes do mesmo estar com completo, em outras palavras, antes de toda mensagem enviada pelo servidor estar escrita no buffer então, o número de bytes das mensagens mostra-se necessário, uma vez que, podemos controlar a função, junto ao envio de mensagens de controle (ACK\footnote{Veja tcp.h[\ref{itm:tcp}]}), afim de ler o buffer apenas quando o mesmo estiver completo.\footnote{The receive calls normally return any data available, up to the requested amount, rather than waiting for receipt of the full amount requested.}
\newpage
\begin{figure}[!htb]
  \centering
  \subfloat[Fluxograma Servidor]{
  \includegraphics[width=150px,height=300px]{fluxo_servidor.png}
  \label{fluxoservidor}
  }
  \quad %espaco separador
   \subfloat[Fluxograma Cliente]{
     \label{fluxoserver}
     \includegraphics[width=150px,height=300px]{fluxo_cliente.png}
   }
   \caption{Fluxogramas}
   \label{figfluxos}
\end{figure}
\newpage
\begin{figure}[!htb]
  \centering
  \label{fluxotempo}
  \includegraphics[scale=0.5]{fluxo_tempo.png}
  \caption{Definição do cálculo dos tempos}
\end{figure}
\newpage
\section{Resultados e discussões}
Os testes foram efetuados em duas máquinas, ambas conectadas à rede porém, não localmente. Denominaremos a máquina servidor como [S] e a cliente como [C]. [S] e [C] estão em continentes diferentes.
\\Foram efetuadas 100 medições para cada opção do menu[\ref{itm:tcpserver}], ao todo foram 600 medições de tempo. Dividiu-se o tempo em tempo de processamento e tempo de comunicação, sendo o último a diferença entre o tempo total e o tempo de processamento.
\\Abaixo demostramos o traceroute, o qual mostra a rota percorrida por um pequeno pacote, marcando assim os roteadores por qual passa. Vale ressaltar que os teste 
foram realizados em uma rede sem fio, sendo assim há uma perda maior de pacotes, e quanto maior a rota a ser percorrida, também espera-se uma perda maior.
\lstinputlisting{traceroute}
\subsection{Tabelas e gráficos}
\subsubsection{Tempo total}[\ref{fluxotempo}]
Representaremos todas as 100 medidas das 6 opções do menu[\ref{itm:tcpserver}]
\begin{table}
  \tiny
  \centering
  \begin{tabular}{|c|}
    \hline
    Opção 1 [ms] \\
    \hline
    186134.000000\\
187389.000000\\
187232.000000\\
186872.000000\\
184125.000000\\
186449.000000\\
186252.000000\\
189478.000000\\
183976.000000\\
208636.000000\\
201516.000000\\
208826.000000\\
199975.000000\\
187522.000000\\
187232.000000\\
186570.000000\\
184801.000000\\
187593.000000\\
187224.000000\\
199953.000000\\
184321.000000\\
184627.000000\\
185264.000000\\
182234.000000\\
185997.000000\\
183630.000000\\
195198.000000\\
185220.000000\\
184872.000000\\
184440.000000\\
215691.000000\\
185690.000000\\
185212.000000\\
184893.000000\\
184262.000000\\
187784.000000\\
190727.000000\\
186847.000000\\
187853.000000\\
187898.000000\\
184341.000000\\
192566.000000\\
192088.000000\\
195817.000000\\
189500.000000\\
187443.000000\\
197499.000000\\
188974.000000\\
187302.000000\\
187996.000000\\
207983.000000\\
183661.000000\\
188076.000000\\
193033.000000\\
203251.000000\\
191111.000000\\
208720.000000\\
191228.000000\\
208735.000000\\
192248.000000\\
188302.000000\\
199366.000000\\
185200.000000\\
189354.000000\\
184946.000000\\
212453.000000\\
186990.000000\\
187158.000000\\
186925.000000\\
187869.000000\\
190985.000000\\
192168.000000\\
188490.000000\\
198904.000000\\
211250.000000\\
190856.000000\\
186611.000000\\
184224.000000\\
188581.000000\\
200660.000000\\
187730.000000\\
183613.000000\\
182749.000000\\
184443.000000\\
191332.000000\\
186974.000000\\
183429.000000\\
188809.000000\\
187492.000000\\
189442.000000\\
188319.000000\\
212336.000000\\
184941.000000\\
199550.000000\\
188052.000000\\
184290.000000\\
189488.000000\\
187303.000000\\
187554.000000\\
184799.000000\\
    
    \hline
  \end{tabular}
  \begin{tabular}{|c|}
    \hline
    Opção 2 [ms] \\
    \hline
    201859.0\\
202306.817001\\
199848.501998\\
201010.160995\\
199215.401000\\
208501.749000\\
201476.560997\\
198648.371002\\
198738.604995\\
198555.204002\\
202348.875\\
204144.968002\\
223860.324996\\
202700.039993\\
193195.859001\\
199188.389999\\
203041.983001\\
214173.309997\\
216054.957000\\
197666.222000\\
196235.888999\\
207876.913002\\
199879.262001\\
198508.363998\\
207005.726997\\
210798.0\\
210775.443000\\
203899.921005\\
204159.311996\\
200577.259994\\
195458.647994\\
221667.504005\\
209850.762001\\
198213.652999\\
211319.792999\\
209023.158004\\
210316.257003\\
216141.311004\\
205385.445999\\
209093.493995\\
225615.328002\\
198674.750999\\
195472.144004\\
205143.945999\\
208333.218002\\
195988.058998\\
198926.396003\\
202343.136001\\
203655.444000\\
209972.173004\\
200545.443000\\
197275.022003\\
200417.637001\\
219973.697998\\
200157.276000\\
220715.338005\\
207236.249000\\
196315.908996\\
197564.450996\\
200287.746002\\
212831.952003\\
307414.807998\\
240003.151000\\
208634.329002\\
200791.722999\\
216155.674003\\
201785.068000\\
195292.268997\\
195221.592002\\
200974.921005\\
212591.645996\\
195282.427001\\
198348.010993\\
197866.603004\\
213719.018005\\
202181.004997\\
199644.747001\\
197390.852996\\
201281.739997\\
202089.450004\\
211811.261001\\
196298.339996\\
220099.944000\\
197679.093002\\
211163.772003\\
213207.867996\\
219191.463996\\
211865.255996\\
200830.794006\\
210114.912002\\
198390.705993\\
205560.611999\\
205478.828994\\
203360.357002\\
199114.817001\\
201293.845001\\
207830.158004\\
197698.594001\\
203239.609001\\
197611.225997\\

    \hline
  \end{tabular}
  \begin{tabular}{|c|}
    \hline
    Opção 3 [ms] \\
    \hline
    203455.275001\\
211206.041999\\
197908.959999\\
202196.333999\\
198711.988998\\
200473.372001\\
208630.888000\\
207026.652000\\
218691.422004\\
207706.132995\\
222586.570999\\
197973.020004\\
211164.223999\\
225928.515998\\
198874.804000\\
198840.096000\\
205587.041999\\
207217.750999\\
208344.699996\\
202186.023002\\
196298.675003\\
199305.034004\\
195543.421997\\
211049.842002\\
201241.436996\\
196029.867996\\
213466.261001\\
197755.699996\\
215781.550003\\
202633.296997\\
207692.336997\\
199085.402000\\
209131.826995\\
199962.319999\\
200131.684997\\
211305.824996\\
199024.714996\\
199656.143997\\
200330.039993\\
235347.397994\\
314829.455001\\
197507.366004\\
195592.687995\\
198117.076995\\
218314.411003\\
197601.516998\\
696983.492996\\
201467.684997\\
222263.551002\\
308753.834999\\
199199.266006\\
200401.573997\\
913252.808998\\
245864.714996\\
196003.160995\\
197564.656997\\
211258.578002\\
203892.424995\\
211127.557998\\
204189.597999\\
209724.394996\\
210252.112998\\
211643.968002\\
201642.031997\\
197560.337997\\
213216.848999\\
212245.299995\\
197395.237998\\
196868.642997\\
200111.273994\\
219745.275001\\
212442.816001\\
212264.862998\\
196730.083000\\
226795.617004\\
199681.481994\\
196140.077003\\
215272.972999\\
213744.094001\\
212239.170997\\
211371.820999\\
201663.841995\\
208548.171997\\
199326.139999\\
208015.150001\\
197828.065002\\
226185.018997\\
226055.114997\\
199736.810997\\
202952.624000\\
198478.914001\\
199884.455001\\
198851.379997\\
198868.838996\\
218355.901000\\
197110.048004\\
195512.235000\\
224791.856994\\
200772.836997\\
199104.977996\\

    \hline
  \end{tabular}
  \begin{tabular}{|c|}
    \hline
    Opção 4 [ms] \\
    \hline
    387384.000000\\
374474.000000\\
427843.000000\\
382172.000000\\
378222.000000\\
376055.000000\\
376315.000000\\
194417.000000\\
380161.000000\\
379678.000000\\
405222.000000\\
388630.000000\\
378038.000000\\
372268.000000\\
382632.000000\\
197574.000000\\
580704.000000\\
380162.000000\\
404885.000000\\
385934.000000\\
377787.000000\\
381415.000000\\
648557.000000\\
374264.000000\\
388695.000000\\
384329.000000\\
207618.000000\\
379876.000000\\
376061.000000\\
213692.000000\\
376815.000000\\
226705.000000\\
219818.000000\\
557030.000000\\
373069.000000\\
382635.000000\\
382833.000000\\
384977.000000\\
383732.000000\\
391522.000000\\
207540.000000\\
574534.000000\\
375237.000000\\
195837.000000\\
600420.000000\\
377294.000000\\
195515.000000\\
378985.000000\\
379188.000000\\
208805.000000\\
570363.000000\\
372253.000000\\
211366.000000\\
388508.000000\\
392704.000000\\
495126.000000\\
407102.000000\\
211494.000000\\
384187.000000\\
382093.000000\\
211832.000000\\
372351.000000\\
375555.000000\\
572923.000000\\
386961.000000\\
386677.000000\\
392726.000000\\
380430.000000\\
384439.000000\\
604662.000000\\
389692.000000\\
398815.000000\\
562676.000000\\
372192.000000\\
375258.000000\\
196175.000000\\
418567.000000\\
372079.000000\\
194860.000000\\
422544.000000\\
377544.000000\\
374463.000000\\
376970.000000\\
376326.000000\\
400862.000000\\
391890.000000\\
206752.000000\\
412661.000000\\
390881.000000\\
223366.000000\\
373351.000000\\
400206.000000\\
575655.000000\\
388674.000000\\
383672.000000\\
198639.000000\\
377000.000000\\
372877.000000\\
387913.000000\\
435362.000000\\

    \hline
  \end{tabular}
  \begin{tabular}{|c|}
    \hline
    Opção 5 [ms] \\
    \hline
    798799.000000\\
809658.000000\\
806619.000000\\
807654.000000\\
841950.000000\\
804294.000000\\
794001.000000\\
803674.000000\\
823421.000000\\
836823.000000\\
812663.000000\\
827006.000000\\
802566.000000\\
200806.000000\\
791945.000000\\
810600.000000\\
829574.000000\\
819217.000000\\
806333.000000\\
790672.000000\\
815474.000000\\
860111.000000\\
796783.000000\\
875436.000000\\
903562.000000\\
799664.000000\\
805160.000000\\
877402.000000\\
802668.000000\\
793100.000000\\
819994.000000\\
790226.000000\\
789644.000000\\
803754.000000\\
794327.000000\\
793979.000000\\
810330.000000\\
841225.000000\\
818517.000000\\
799198.000000\\
789596.000000\\
787954.000000\\
821004.000000\\
799028.000000\\
785814.000000\\
1103251.000000\\
808306.000000\\
798444.000000\\
840483.000000\\
805219.000000\\
812946.000000\\
1204824.000000\\
803446.000000\\
785016.000000\\
801234.000000\\
810304.000000\\
787603.000000\\
816566.000000\\
803445.000000\\
822669.000000\\
789270.000000\\
788115.000000\\
785006.000000\\
808844.000000\\
796992.000000\\
824105.000000\\
788755.000000\\
794453.000000\\
811171.000000\\
799804.000000\\
799161.000000\\
787083.000000\\
789392.000000\\
795363.000000\\
788889.000000\\
801048.000000\\
848622.000000\\
820199.000000\\
796829.000000\\
802407.000000\\
813252.000000\\
795081.000000\\
818911.000000\\
829892.000000\\
803175.000000\\
801467.000000\\
785462.000000\\
820895.000000\\
880082.000000\\
791788.000000\\
788466.000000\\
817210.000000\\
828234.000000\\
788479.000000\\
791639.000000\\
836845.000000\\
786796.000000\\
802906.000000\\
788881.000000\\
801990.000000\\

    \hline
  \end{tabular}
  \begin{tabular}{|c|}
    \hline
    Opção 6 [ms] \\
    \hline
    206021.424995\\
221135.761001\\
194634.906005\\
208775.874000\\
227648.604003\\
201682.800003\\
209009.982002\\
196474.446006\\
210193.853996\\
210716.194000\\
196513.631004\\
198629.799995\\
208428.291000\\
212321.134994\\
200682.558006\\
201131.635002\\
222372.802001\\
199312.712005\\
196317.111999\\
196014.212997\\
201738.277999\\
198154.059997\\
207953.881996\\
197331.064002\\
198204.980003\\
197909.973999\\
194491.701995\\
211310.005996\\
196021.000999\\
211372.725997\\
228070.555999\\
195635.835998\\
197955.312004\\
196819.007003\\
194700.744995\\
196085.821998\\
202845.762001\\
209316.425003\\
230620.589004\\
201132.019004\\
225474.807998\\
195188.798995\\
198957.187004\\
205883.418998\\
194414.484001\\
238461.251998\\
206435.582000\\
199757.081001\\
213866.972000\\
225913.565002\\
196546.166000\\
213463.575996\\
195959.438003\\
206839.480995\\
220540.803001\\
209264.978004\\
197839.481994\\
392216.760002\\
199549.319000\\
197799.188003\\
203430.643997\\
208214.865997\\
213407.930999\\
202921.930000\\
215385.958999\\
220633.361999\\
197105.193000\\
197583.671005\\
192261.052001\\
194151.289001\\
215815.277999\\
194891.460998\\
201080.104003\\
198913.440002\\
202933.259002\\
198982.377998\\
195121.241996\\
202507.513999\\
197895.453002\\
199605.223999\\
210042.898002\\
197364.318000\\
198122.929000\\
195653.922996\\
210346.429000\\
215784.079002\\
210709.113998\\
288172.521995\\
207316.75\\
204968.500999\\
196863.434997\\
200143.079002\\
217198.010002\\
208987.342994\\
207090.613998\\
194524.899002\\
199193.183006\\
221173.065002\\
195255.782997\\
195518.802001\\

    \hline
  \end{tabular}
  \caption{Tabela de tempo total}
\end{table}
\newpage
\subsubsection{Tempo de processamento}[\ref{fluxotempo}]
Representaremos todas as 100 medidas das 6 opções do menu[\ref{itm:tcpserver}]
\begin{table}
  \tiny
  \centering
  \begin{tabular}{|c|}
    \hline
    Opção 1 [ms] \\
    \hline
    23499.42675\\
2741.711914\\
2426.067382\\
2293.096679\\
2483.584960\\
2732.129882\\
2496.462890\\
2700.336914\\
2766.775390\\
2358.172851\\
2197.652343\\
2110.012695\\
1998.833007\\
2033.600585\\
15533.80859\\
3161.451171\\
2274.647460\\
2100.997070\\
2092.358398\\
2376.628906\\
2383.828125\\
2089.916992\\
2098.202148\\
2298.115234\\
2955.897460\\
2044.863281\\
1960.921875\\
2085.931640\\
1999.516601\\
1985.002929\\
2314.264648\\
3211.170898\\
2230.250976\\
2409.091796\\
2142.537109\\
2088.868164\\
2257.673828\\
2641.204101\\
2255.297851\\
1825.700195\\
2079.362304\\
3437.389648\\
1926.607421\\
1862.973632\\
1717.3359375\\
2198.416015\\
2468.734375\\
2300.391601\\
2601.793945\\
1718.525390\\
1825.108398\\
1918.760742\\
1763.375\\
1587.811523\\
4070.6875\\
1531.560546\\
1527.393554\\
1513.079101\\
1525.045898\\
1734.748046\\
1660.745117\\
1561.499023\\
1903.579101\\
1511.046875\\
2468.661132\\
1588.126953\\
1541.253906\\
1734.319335\\
1748.565429\\
1281.558593\\
1318.267578\\
1490.784179\\
1541.271484\\
1308.565429\\
1313.391601\\
1450.140625\\
1223.756835\\
1944.65625\\
1305.014648\\
1267.692382\\
1641.869140\\
2623.924804\\
1224.431640\\
1249.188476\\
1685.751953\\
1371.079101\\
1380.0625\\
2513.627929\\
1259.575195\\
1453.826171\\
1326.1875\\
1325.004882\\
2211.475585\\
1201.508789\\
1698.415039\\
1963.515625\\
1187.314453\\
1281.374023\\
1325.030273\\
1306.115234\\

    \hline
  \end{tabular}
  \begin{tabular}{|c|}
    \hline
    Opção 2 [ms] \\
    \hline
    2588.69042\\
2166.27148\\
2984.88476\\
2306.31054\\
2287.46679\\
3056.28710\\
2742.53710\\
2058.77929\\
2033.60839\\
2092.90527\\
2037.83398\\
2029.88183\\
2053.88671\\
1992.09375\\
1939.20410\\
1906.79589\\
1929.55371\\
1996.64062\\
2156.50292\\
1977.66015\\
2045.03320\\
2102.73828\\
2213.47363\\
1944.46582\\
2150.33398\\
1784.53515\\
2122.51562\\
2024.09765\\
1937.89843\\
2206.93847\\
2116.26074\\
2156.47656\\
1826.13378\\
1969.08007\\
1745.32519\\
1863.08691\\
1610.82519\\
2063.41601\\
2386.09082\\
1649.90136\\
2705.86132\\
1729.25097\\
1865.64257\\
1663.25292\\
1903.49804\\
2441.68457\\
1757.96777\\
1627.75585\\
1736.50781\\
2048.62695\\
2008.40332\\
1573.51953\\
2080.39160\\
1725.84277\\
1640.64648\\
1443.74121\\
1933.15820\\
2116.26660\\
1625.32910\\
1714.30273\\
1802.28710\\
1583.49414\\
1719.99902\\
1610.22070\\
1849.58886\\
1530.31738\\
2223.86230\\
1616.85156\\
1421.30468\\
2223.39160\\
1423.40820\\
1587.52441\\
3861.52050\\
1734.07714\\
1808.31738\\
1615.26269\\
2609.58300\\
1638.81835\\
1663.13281\\
1699.91308\\
2133.82617\\
1397.0\\
1598.37109\\
1326.52441\\
1423.82812\\
1680.19140\\
1429.39355\\
1819.98437\\
1721.93847\\
1437.08105\\
1474.50097\\
1431.42285\\
1599.43457\\
1927.44531\\
1943.12304\\
1764.84277\\
1490.30273\\
1784.49218\\
1686.62794\\
1413.28906\\

    \hline
  \end{tabular}
  \begin{tabular}{|c|}
    \hline
    Opção 3 [ms] \\
    \hline
    3674.17187\\
2809.39257\\
2807.36425\\
2515.47070\\
2921.58496\\
3067.58593\\
2541.833984\\
2312.737304\\
2513.616210\\
2345.302734\\
2515.03125\\
2288.631835\\
2425.683593\\
2267.762695\\
2315.449218\\
2256.185546\\
2201.988281\\
2189.830078\\
2962.610351\\
2055.752929\\
3069.470703\\
2143.847656\\
2408.840820\\
2326.465820\\
2477.500976\\
2418.101562\\
4408.851562\\
2209.492187\\
2048.783203\\
3774.604492\\
2004.117187\\
2419.379882\\
2079.896484\\
6134.481445\\
1894.795898\\
2232.594726\\
2940.666992\\
1885.816403\\
1824.250976\\
2054.471679\\
1824.631835\\
2103.592773\\
2128.505859\\
1940.451171\\
2138.087890\\
1994.018554\\
2152.935546\\
2364.742187\\
4888.248046\\
2240.315429\\
2052.795898\\
3819.780273\\
2134.796875\\
1930.450195\\
1781.019531\\
1852.238281\\
1853.941406\\
1934.556640\\
1688.001953\\
4222.914062\\
2009.717773\\
2002.026367\\
1904.137695\\
1703.552734\\
1850.693359\\
1758.201171\\
1747.850585\\
1685.582031\\
2361.884765\\
1656.470703\\
1749.207031\\
1746.084960\\
2101.828125\\
1789.326171\\
2061.145507\\
1625.257812\\
1787.131835\\
1682.167968\\
1763.418945\\
1989.556640\\
2044.693359\\
2691.419921\\
1602.276367\\
1600.141601\\
1648.862304\\
1390.353515\\
1708.804687\\
1545.240234\\
2186.770507\\
1739.717773\\
1959.007812\\
1788.939453\\
1715.9375\\
2120.757812\\
1583.235351\\
1576.397460\\
1949.586914\\
1911.683593\\
1786.504882\\
2769.505859\\

    \hline
  \end{tabular}
  \begin{tabular}{|c|}
    \hline
    Opção 4 [ms] \\
    \hline
    555.000000\\
545.000000\\
554.000000\\
577.000000\\
550.000000\\
547.000000\\
543.000000\\
550.000000\\
569.000000\\
545.000000\\
560.000000\\
569.000000\\
552.000000\\
531.000000\\
543.000000\\
541.000000\\
536.000000\\
561.000000\\
545.000000\\
543.000000\\
550.000000\\
554.000000\\
556.000000\\
551.000000\\
563.000000\\
581.000000\\
586.000000\\
565.000000\\
548.000000\\
551.000000\\
575.000000\\
552.000000\\
557.000000\\
547.000000\\
553.000000\\
556.000000\\
555.000000\\
543.000000\\
591.000000\\
582.000000\\
546.000000\\
545.000000\\
581.000000\\
550.000000\\
556.000000\\
596.000000\\
542.000000\\
551.000000\\
591.000000\\
549.000000\\
549.000000\\
547.000000\\
565.000000\\
567.000000\\
570.000000\\
622.000000\\
547.000000\\
553.000000\\
587.000000\\
570.000000\\
555.000000\\
553.000000\\
542.000000\\
551.000000\\
565.000000\\
551.000000\\
559.000000\\
560.000000\\
570.000000\\
547.000000\\
561.000000\\
557.000000\\
561.000000\\
558.000000\\
540.000000\\
566.000000\\
531.000000\\
545.000000\\
576.000000\\
562.000000\\
552.000000\\
560.000000\\
589.000000\\
548.000000\\
546.000000\\
546.000000\\
554.000000\\
573.000000\\
550.000000\\
552.000000\\
555.000000\\
566.000000\\
549.000000\\
558.000000\\
557.000000\\
600.000000\\
553.000000\\
545.000000\\
546.000000\\
547.000000\\

    \hline
  \end{tabular}
  \begin{tabular}{|c|}
    \hline
    Opção 5 [ms] \\
    \hline
    587293.000000\\
564101.000000\\
568421.000000\\
555928.000000\\
572451.000000\\
556187.000000\\
564367.000000\\
564026.000000\\
600243.000000\\
564148.000000\\
560172.000000\\
560065.000000\\
588025.000000\\
568018.000000\\
560106.000000\\
582428.000000\\
591891.000000\\
571918.000000\\
559977.000000\\
568044.000000\\
576041.000000\\
560274.000000\\
560677.000000\\
556033.000000\\
572326.000000\\
560592.000000\\
564051.000000\\
587935.000000\\
558832.000000\\
564474.000000\\
563958.000000\\
604257.000000\\
580073.000000\\
596047.000000\\
584255.000000\\
592302.000000\\
588301.000000\\
604191.000000\\
576113.000000\\
564300.000000\\
604714.000000\\
569860.000000\\
555921.000000\\
572221.000000\\
559985.000000\\
575807.000000\\
572493.000000\\
559996.000000\\
576394.000000\\
604193.000000\\
572449.000000\\
559829.000000\\
567938.000000\\
556361.000000\\
573747.000000\\
620089.000000\\
579521.000000\\
585557.000000\\
559933.000000\\
562390.000000\\
560382.000000\\
564387.000000\\
592450.000000\\
583946.000000\\
604141.000000\\
569690.000000\\
580572.000000\\
576014.000000\\
560146.000000\\
562996.000000\\
564238.000000\\
560074.000000\\
587962.000000\\
556263.000000\\
588097.000000\\
564699.000000\\
553780.000000\\
556397.000000\\
563761.000000\\
592242.000000\\
568260.000000\\
564199.000000\\
594113.000000\\
568013.000000\\
573846.000000\\
580129.000000\\
564235.000000\\
555879.000000\\
552448.000000\\
564098.000000\\
553049.000000\\
569645.000000\\
556187.000000\\
564195.000000\\
568313.000000\\
556060.000000\\
571894.000000\\
561857.000000\\
572339.000000\\
569532.000000\\

    \hline
  \end{tabular}
  \begin{tabular}{|c|}
    \hline
    Opção 6 [ms] \\
    \hline
    206021.424995\\
221135.761001\\
194634.906005\\
208775.874000\\
227648.604003\\
201682.800003\\
209009.982002\\
196474.446006\\
210193.853996\\
210716.194000\\
196513.631004\\
198629.799995\\
208428.291000\\
212321.134994\\
200682.558006\\
201131.635002\\
222372.802001\\
199312.712005\\
196317.111999\\
196014.212997\\
201738.277999\\
198154.059997\\
207953.881996\\
197331.064002\\
198204.980003\\
197909.973999\\
194491.701995\\
211310.005996\\
196021.000999\\
211372.725997\\
228070.555999\\
195635.835998\\
197955.312004\\
196819.007003\\
194700.744995\\
196085.821998\\
202845.762001\\
209316.425003\\
230620.589004\\
201132.019004\\
225474.807998\\
195188.798995\\
198957.187004\\
205883.418998\\
194414.484001\\
238461.251998\\
206435.582000\\
199757.081001\\
213866.972000\\
225913.565002\\
196546.166000\\
213463.575996\\
195959.438003\\
206839.480995\\
220540.803001\\
209264.978004\\
197839.481994\\
392216.760002\\
199549.319000\\
197799.188003\\
203430.643997\\
208214.865997\\
213407.930999\\
202921.930000\\
215385.958999\\
220633.361999\\
197105.193000\\
197583.671005\\
192261.052001\\
194151.289001\\
215815.277999\\
194891.460998\\
201080.104003\\
198913.440002\\
202933.259002\\
198982.377998\\
195121.241996\\
202507.513999\\
197895.453002\\
199605.223999\\
210042.898002\\
197364.318000\\
198122.929000\\
195653.922996\\
210346.429000\\
215784.079002\\
210709.113998\\
288172.521995\\
207316.75\\
204968.500999\\
196863.434997\\
200143.079002\\
217198.010002\\
208987.342994\\
207090.613998\\
194524.899002\\
199193.183006\\
221173.065002\\
195255.782997\\
195518.802001\\

    \hline
  \end{tabular}
  \caption{Tabela de tempo de processamento}
\end{table}
\newpage
\subsection{Médias e gráficos}
Devido ao grande número de dados decidimos por representar graficamente apenas as
médias e desvios de cada opção do menu[\ref{itm:tcpserver}].\\
\subsubsection{Cálculos}
A média calculada entre os pontos foi a média simples, em outras palavras, denominamos a 
média como \boxed{\mu} então, \boxed{\mu = \frac{1}{N}\sum_{i=1}^{N}\left(x_i\right)}, onde N
é o número de pontos, no caso desse projeto temos N=100 medidas para cada uma das opções de menu[\ref{itm:tcpserver}].\\
Já o desvio padrão amostral, denominado \boxed{\sigma} é calculado como \boxed{\sigma = \sqrt{\frac{1}{N}\sum_{i=1}^{N}\left(x_i - \mu\right)^2}}\\
Seja \boxed{\mu_t}, \boxed{\mu_p}, \boxed{\mu_c}, a média do tempo total, média do tempo de processamento e média do tempo de comunicação,
respectivamente, e o desvio padrão do tempo de comunicação, de
nominado \boxed{\sigma_c}.
\newpage
\begin{figure}[!htb]
  \centering
  \subfloat[Gráficos Tempo de comunicação e total]{
  \includegraphics[width=300px,height=500px]{graficos.png}
  \label{graficos}
  }
\end{figure}
\newpage
\begin{table}
  \centering
  \begin{tabular}{|c|c|c|c||c|}
    \hline
    Menu & \boxed{\mu_t} & \boxed{\mu_p} & \boxed{\mu_c} &  \boxed{\pm \sigma_c} \\
    \hline
    1 & 206212,67 & 32,82 & 206179,85 & 41671.14 (20,21\%)\\
    2 & 401828,39 & 202106,08 & 199722,31  & 13750.53 (6,88\%)\\
    3 & 403070,83 & 200304,75 & 202766,08 & 19338.60 (9,54\%)\\
    4 & 228492,74 & 537,03 & 227955.71 & 74957.25 (32,88\%)\\
    5 & 809829,40 & 599657,18 & 210172.22 & 19006.89 (9,04\%)\\
    6 & 400696,12 & 201315,01 & 199381.11 & 19006.89 (8,55\%)\\
    \hline
  \end{tabular}
  \caption{Tabela com médias e desvios para protocolo UDP}
  \centering
  \begin{tabular}{|c|c|c|c||c|}
    \hline
    Menu & \boxed{\mu_t} & \boxed{\mu_p} & \boxed{\mu_c} &  \boxed{\pm \sigma_c} \\
    \hline
    1 & 229928.97 & 59.43 & 229869.54 & 80776.07 (35,13\%)\\
    2 & 477881.95 & 236349.80 & 241532.15  & 94761.02 (39.23\%)\\
    3 & 489242.44 & 245704.79 & 243537.65 & 70828.31 (29,08\%)\\
    4 & 494792.56 & 559.17 & 494233.39 & 305558.45 (61,82\%)\\
    5 & 1027348.23 & 727262.31 & 300085.92 & 77658.26 (25,87\%)\\
    6 & 458069.60 & 232573.16 & 225496.44 & 61852.36 (30.09\%)\\
    \hline
  \end{tabular}
  \caption{Tabela com médias e desvios para protocolo TCP}
\end{table}
\newpage
\section{Conclusão}
Visto a simplicidade do protocolo UDP, tal como ausência de mecanismos de controle de fluxo, confiabilidade da entrega do pacote e o uso de datagrams, ao invez de 
streams, utilizados no protocolo TCP, espera-se um tempo de comunicação menor. Os dados do sistema com UDP não foram comparados com os dados obtidos com o TCP no 
relatório anterior, e sim novos testes para esse último foram realizados. Como previsto, o procotolo UDP teve um tempo de comunicação menor, chegando a uma 
diferença de 46.12\% no caso da opção 4 do menu (envio de apenas uma requisição por parte do cliente), 70.03\% opção 5 (envio de 3 requisições, e 4 respostas entre cliente/servidor). Nota-se que os desvio padrão de ambos os protocolos mas, principalmente do TCP, são altos, isso ocorre, provavelmente, a uma grande instabilidade
da rede, uma vez que utiliza-se rede sem fio (há uma perda considerável de pacotes, sendo que o TCP garante a entrega e não duplicada) e também, pelo mesmo motivo 
que os testes, para o TCP, foram refeitos, a rede no dia dos testes apresentava-se lenta e com falhas na comunicação para qualquer aplicação que a utiliza-se.
Portanto, concluimos que o protocolo UDP, realmente, é indicado para aplicações que necessitem de velocidade e simplicidade na comunicação entre cliente e servidor.
Já sistemas que requerem confiabilidade, tal como garantia da integridade e entrega da informação, o protocolo TCP seria o mais indicado.
\newpage
\section{Código fonte \textit{(.c)}}
\lstset{ 
   basicstyle=\footnotesize,        % the size of the fonts that are used for the code
   breaklines=true,                 % sets automatic line breaking
   captionpos=b,                    % sets the caption-position to bottom
   frame=single,                    % adds a frame around the code
   language=C,		                 % the language of the code
   numbers=left,                    % where to put the line-numbers; possible values are (none, left, right)
   numbersep=5pt,                   % how far the line-numbers are from the code
   rulecolor=\color{black},         % if not set, the frame-color may be changed on line-breaks within not-black text (e.g. comments (green here))
   showspaces=false,                % show spaces everywhere adding particular underscores; it overrides 'showstringspaces'
   showstringspaces=false,          % underline spaces within strings only
   showtabs=false,                  % show tabs within strings adding particular underscores
   tabsize=2,                       % sets default tabsize to 2 spaces
}
\subsection{Diretório \textit{commons}}
\subsubsection{\label{itm:error.c}\emph{error.h error.c}}
\lstinputlisting{../../tcp/common/error.h}
\lstinputlisting{../../tcp/common/error.c}
\subsubsection{\label{itm:common.c}\emph{commmon.h common.c}}
\lstinputlisting{../../tcp/common/common.h}
\lstinputlisting{../../tcp/common/common.c}
\subsubsection{\label{itm:avl.c}\emph{avl.h avl.c}}
\lstinputlisting{../../tcp/common/avl.h}
\lstinputlisting{../../tcp/common/avl.c}
\subsubsection{\label{itm:archives.c}\emph{archives.h archives.c}}
\lstinputlisting{../../tcp/common/archives.h}
\lstinputlisting{../../tcp/common/archives.c}
\subsubsection{\label{itm:books.c}\emph{books.h books.c}}
\lstinputlisting{../../tcp/common/books.h}
\lstinputlisting{../../tcp/common/books.c}
\subsubsection{\label{itm:tcp.c}\emph{tcp.h tcp.c}}
\lstinputlisting{../../tcp/common/tcp.h}
\lstinputlisting{../../tcp/common/tcp.c}
\subsubsection{\label{itm:tempo.c}\emph{tempo.h tempo.c}}
\lstinputlisting{../../tcp/common/tempo.h}
\lstinputlisting{../../tcp/common/tempo.c}
\subsection{Diretório \textit{server}}
\subsubsection{\label{itm:server.c}\emph{server.h server.c}}
\lstinputlisting{../server/server.h}
\lstinputlisting{../server/server.c}
\subsubsection{\label{itm:udpserver.c}\emph{udpserver.h udpserver.c}}
\lstinputlisting{../server/udp_server.h}
\lstinputlisting{../server/udp_server.c}
\subsubsection{\label{itm:login.c}\emph{login.h login.c}}
\lstinputlisting{../server/login.h}
\lstinputlisting{../server/login.c}
\subsection{Diretório \textit{client}}
\subsubsection{\label{itm:client.c}\emph{client.h client.c}}
\lstinputlisting{../client/client.h}
\lstinputlisting{../client/client.c}
\subsubsection{\label{itm:udpclient.c}\emph{udpclient.h udpclient.c}}
\lstinputlisting{../client/udp_client.h}
\lstinputlisting{../client/udp_client.c}
\end{document}
