%*************************************************

\documentclass[a4paper,10pt]{article}
\usepackage[brazil]{babel}
\usepackage[utf8]{inputenc}
%\usepackage[latin1]{inputenc}
\usepackage{amsthm,amsfonts,amsmath,amssymb}
%img
\usepackage{graphicx}
\usepackage{subfig}
%tabela

%fim
%fim
\usepackage{makeidx}
\usepackage{enumerate}
\usepackage{hyperref}
\hypersetup{
  colorlinks,
  linkcolor=blue,
  filecolor=blue,
  urlcolor=blue,
  citecolor=blue 
}
%titulo
\title{Servidor Concorrente TCP}
\author{Leandro Kümmel Tria Mendes RA033910 \\ Fernando Teixeira RA??????}
\makeindex
%inicio
\begin{document}
\maketitle
\tableofcontents
\listoffigures
\listoftables
\section{Introdução}
O objetivo desse projeto é implementar um sistema cliente/servidor
concorrente, com operações para o gerenciamento de livros em uma livraria.
Na comunicação entre cliente e servidor utilizou-se o protocolo TCP\footnote{
Mais informações sobre TCP \url{http://www.linktionary.com/t/tcp.html} }, da camada de transporte, e a partir da execução de testes podemos avaliar alguns aspectos
desse protocolo e posteriormente compará-lo com outro(s).
\section{Desenvolvimento}
Linux foi o sistema operacional utilizado para o desenvolvimento (distribuição
2.6.43.8-1.fc15.i686). Igualmente, para os testes utilizou-se duas máquinas com linux, porém com distribuições diferentes.
\subsection{Protocolo TCP - Transmission Control Protocol}
O protocolo TCP foi escrito de modo a garantir que os dados enviados (pelo
servido) e recebidos (pelo cliente) de forma correta, na sequência 
adequada e sem erros, pela rede.
\\As características fundamentais do TCP são:
\begin{itemize}
\item \emph{Orientado à conexão}: necessidade de uma conexão.
\item \emph{Ponto a ponto}: conexão é estabelecida entre dois pontos.
\item \emph{Confiabilidade}: Permite a recuperação de arquivos perdidos, elimina 
arquivos duplicados, recupera dados corrompidos, entrega na ordem do envio e 
pode recuperar o \"link\" entre cliente e servidor, caso esse, por algum motivo,
seja perdido.
\item \emph{Full duplex}: Possível transferência simultânea, entre cliente e 
servidor
\item \emph{Handshake}: Mecanismo de estabelecimento e finalização de conexão.
O TCP garante que, no final da conexão, \textbf{todos} os pacotes sejam entregues
/recebidos
\item \emph{Entrega ordenada}: A aplicação entrega ao TCP bloco de dados de 
tamanho variável. Esse protocolo divide estes dados em segmentos de tamanho 
especificado (valor MTU). Sabe-se que camadas inferiores a de transporte podem 
fazer com que os pacotes não cheguem na ordem em que foram enviados. Porém, o TCP
garante a reconstrução dos segmentos no cliente (TCP utiliza um número de 
sequência)
\item \emph{Controle de fluxo}: O protocolo em estudo utiliza-se de um campos 
denominado janela para controlar o fluxo
\end{itemize}
\subsection{Implementação}
O projeto conta com cinco diretórios, cada um com seu Makefile (exceto relatorio e estat), arquivos
principal (main.c e main.h) e um README.md\footnote{Leia esse arquivo antes de
executar o sistema} para instruções adicionais. Há também um Makefile, o qual 
compi
\\Os diretório são:
\begin{enumerate}[I.]
\label{sec:dirs}
\item \label{itm:common} \emph{common}: Contém arquivos de uso comum, tanto pelo servidor quanto 
pelo cliente, inclusive o arquivo que calcula a média dos testes executados.
\item \label{itm:server} \emph{server}: Contém os arquivos que preparam uma porta para esperar
conexões e manipulam o sistema de livraria.
\item \label{itm:client} \emph{client}: Funções que provêm conexão com um servidor, envio das opções
escolhidas pelo cliente e interface para as respostas do sistema de livraria
(servidor).
\item \label{itm:estat} \emph{estat}: Medidas de tempo efetuadas pelo teste.
\item \label{itm:relatorio} \emph{relatorio}
\end{enumerate}
\subsubsection{Manipulação de dados}
Todas as estruturas utilizadas para leitura/escrita dos livros são dinâmicas. \\Arquivos presentes no diretório \emph{common}[\ref{itm:common}]:
\begin{enumerate}[I.]
\item \emph{error.c error.h}: Gerencia erros que eventualmente podem ocorrer.
\item \emph{common.c common.h}: Funções de uso comum.
\item \label{itm:avl} \emph{avl.c avl.h}: Gerencia a estrutura básica da livraria, utiliza-se 
árvore AVL\footnote{ \url{http://pages.cs.wisc.edu/~ealexand/cs367/NOTES/AVL-Trees/index.html} }, pois a busca, inserção/atualização têm complexidade \boxed{O(logN)} , sendo N
o número de elementos na árvore, no caso a quantidade de livros diferentes.
\item \emph{archives.c archives.h}: Manipula arquivos. Faz a leitura do arquivo
da livraria\footnote{Ver README.md para mais detalhes do arquivo da livraria}.
\item \label{itm:tcp} \emph{tcp.c tcp.h}: Contém apenas algumas constantes.
\item \emph{books.c books.h}: Gerencia a estrutura básica de um livro e seus autores.
\item \label{itm:tempo}\emph{tempo.c tempo.h}: Gerencia a estrutura de testes, lê e escreve em arquivos localizados no diretório \emph{estat}[\ref{itm:estat}].
\item \emph{livros/livros}: Arquivo contendo os livros\footnote{Ver README.md para mais detalhes do arquivo da livraria}.
\end{enumerate}
\subsubsection{Conexão Servidor/Cliente}
Compreende dois diretórios \emph{server}[\ref{itm:server}] e \emph{client}[\ref{itm:client}]
\\\textbf{Servidor}:
\begin{enumerate}[I.]
\item \emph{server.c server.h}: Apenas inicia o servidor dada um número de uma porta.
\item \label{itm:tcpserver} \emph{tcp\_server.c tcp\_server.h}: Gerencia tanto as conexões com os clientes quanto a comunicação, em outras palavras, o tcp\_server.c recebe um stream do tcp\_client.c[\ref{itm:tcpclient}] e envia uma resposta adequada ao mesmo.
\item \emph{login.c login.h}: Gerencia o login necessário para editar a quantidade de um livro.
\end{enumerate}
\textbf{Cliente}:
\begin{enumerate}[I.]
\item \emph{client.c client.h}: Apenas inicia a comunicação com um servidor dado o endereço IP e um número de uma porta.
\item \label{itm:tcpclient} \emph{tcp\_client.c tcp\_client.h}: Gerencia tanto 
a criação de uma conexão entre o cliente e servidor quanto a leitura, da entrada dada pelo usuário do sistema, e a comunicação entre hospedeiro e cliente.
\end{enumerate}
\subsection{Coleta e gerência de dados para testes}
Para realizar os testes implementou-se alguns arquivos adicionas[\ref{itm:tempo}]. Uma constante, denominada NUM\_TESTES\footnote{Nesse sistema consideramos NUM\_TESTES igual a 100} , contém o número de testes a serem realizados, ou seja, cada opção do \emph{menu}[\ref{itm:tcpserver}] é executada NUM\_TESTE vezes. Todos os dados são 
salvos no diretório \emph{estat}[\ref{itm:estat}].
\subsection{Vantagens da implementação}
O sistema de livraria é um sistema robusto e com baixa complexidade de tempo.
A escolha da estrutura de árvore avl[\ref{itm:avl}], possibilitou em boa 
performance em questão de tempo de processamento, uma vez que, essa estrutura
mostrou-se eficaz para o problema e tem possuí melhor complexidade de tempo com 
relação a outras estruturas, além de ser de a implementação e manuntenção serem
simples.
\\ Com relação as conexões e comunicações entre cliente/servidor, vale ressaltar que há uma troca de mensagens\footnote{Sabe-se que o TCP envia/recebe streams} inicial entre os dois, na qual o conteúdo do stream
é o número de bytes da maior mensagem possível a ser enviada pelo servidor.
Como a função \boxed{rcv(int sockfd, void *buf, size_t len, int flags)}\footnote{\url{http://linux.die.net/man/2/recv}} começa a ler o buffer, o qual é escrito o stream enviado pelo servidor, antes do mesmo estar com completo, em outras palavras, antes de toda mensagem enviada pelo servidor estar escrita no buffer então, o número de bytes das mensagens mostra-se necessário, uma vez que, podemos controlar a função \boxed{rcv(int sockfd, void *buf, size_t len, int flags)}, junto ao envio de mensagens de controle (ACK\footnote{Veja tcp.h[\ref{itm:tcp}]}), afim de ler o buffer apenas quando o mesmo estiver completo.\footnote{The receive calls normally return any data available, up to the requested amount, rather than waiting for receipt of the full amount requested.}
\begin{figure}[!htb]
  \centering
  \subfloat[Fluxograma Servidor]{
  \includegraphics[width=150px,height=300px]{fluxo_servidor.png}
  \label{fluxoservidor}
  }
  \quad %espaco separador
   \subfloat[Fluxograma Cliente]{
     \label{fluxoserver}
     \includegraphics[width=150px,height=300px]{fluxo_cliente.png}
   }
   \caption{Fluxogramas}
   \label{figfluxos}
\end{figure}
\begin{figure}[!htb]
  \centering
  \label{fluxotempo}
  \includegraphics[scale=0.5]{fluxo_tempo.png}
  \caption{Definição do cálculo dos tempos}
\end{figure}
\section{Resultados e discussões}
Os testes foram efetuados em duas máquinas, ambas conectadas à rede porém, não localmente. Denominaremos a máquina servidor como [S] e a cliente como [C]. [S] e [C] estão em continentes diferentes.
\\Foram efetuadas 100 medições para cada opção do menu[\ref{itm:tcpserver}], ao todo foram 600 medições de tempo. Dividiu-se o tempo em tempo de processamento e tempo de comunicação, sendo o último a diferença entre o tempo total e o tempo de processamento.
\subsection{Tabelas e gráficos}
\subsubsection{Tempo total}
Representaremos todas as 100 medidas das 6 opções do menu[\ref{itm:tcpserver}]
\begin{table}
  \tiny
  \centering
  \begin{tabular}{|c|}
    \hline
    Opção 1 [ms] \\
    \hline
    186134.000000\\
187389.000000\\
187232.000000\\
186872.000000\\
184125.000000\\
186449.000000\\
186252.000000\\
189478.000000\\
183976.000000\\
208636.000000\\
201516.000000\\
208826.000000\\
199975.000000\\
187522.000000\\
187232.000000\\
186570.000000\\
184801.000000\\
187593.000000\\
187224.000000\\
199953.000000\\
184321.000000\\
184627.000000\\
185264.000000\\
182234.000000\\
185997.000000\\
183630.000000\\
195198.000000\\
185220.000000\\
184872.000000\\
184440.000000\\
215691.000000\\
185690.000000\\
185212.000000\\
184893.000000\\
184262.000000\\
187784.000000\\
190727.000000\\
186847.000000\\
187853.000000\\
187898.000000\\
184341.000000\\
192566.000000\\
192088.000000\\
195817.000000\\
189500.000000\\
187443.000000\\
197499.000000\\
188974.000000\\
187302.000000\\
187996.000000\\
207983.000000\\
183661.000000\\
188076.000000\\
193033.000000\\
203251.000000\\
191111.000000\\
208720.000000\\
191228.000000\\
208735.000000\\
192248.000000\\
188302.000000\\
199366.000000\\
185200.000000\\
189354.000000\\
184946.000000\\
212453.000000\\
186990.000000\\
187158.000000\\
186925.000000\\
187869.000000\\
190985.000000\\
192168.000000\\
188490.000000\\
198904.000000\\
211250.000000\\
190856.000000\\
186611.000000\\
184224.000000\\
188581.000000\\
200660.000000\\
187730.000000\\
183613.000000\\
182749.000000\\
184443.000000\\
191332.000000\\
186974.000000\\
183429.000000\\
188809.000000\\
187492.000000\\
189442.000000\\
188319.000000\\
212336.000000\\
184941.000000\\
199550.000000\\
188052.000000\\
184290.000000\\
189488.000000\\
187303.000000\\
187554.000000\\
184799.000000\\

  \end{tabular}
  \begin{tabular}{|c|}
    \hline
    Opção 2 [ms] \\
    \hline
    201859.0\\
202306.817001\\
199848.501998\\
201010.160995\\
199215.401000\\
208501.749000\\
201476.560997\\
198648.371002\\
198738.604995\\
198555.204002\\
202348.875\\
204144.968002\\
223860.324996\\
202700.039993\\
193195.859001\\
199188.389999\\
203041.983001\\
214173.309997\\
216054.957000\\
197666.222000\\
196235.888999\\
207876.913002\\
199879.262001\\
198508.363998\\
207005.726997\\
210798.0\\
210775.443000\\
203899.921005\\
204159.311996\\
200577.259994\\
195458.647994\\
221667.504005\\
209850.762001\\
198213.652999\\
211319.792999\\
209023.158004\\
210316.257003\\
216141.311004\\
205385.445999\\
209093.493995\\
225615.328002\\
198674.750999\\
195472.144004\\
205143.945999\\
208333.218002\\
195988.058998\\
198926.396003\\
202343.136001\\
203655.444000\\
209972.173004\\
200545.443000\\
197275.022003\\
200417.637001\\
219973.697998\\
200157.276000\\
220715.338005\\
207236.249000\\
196315.908996\\
197564.450996\\
200287.746002\\
212831.952003\\
307414.807998\\
240003.151000\\
208634.329002\\
200791.722999\\
216155.674003\\
201785.068000\\
195292.268997\\
195221.592002\\
200974.921005\\
212591.645996\\
195282.427001\\
198348.010993\\
197866.603004\\
213719.018005\\
202181.004997\\
199644.747001\\
197390.852996\\
201281.739997\\
202089.450004\\
211811.261001\\
196298.339996\\
220099.944000\\
197679.093002\\
211163.772003\\
213207.867996\\
219191.463996\\
211865.255996\\
200830.794006\\
210114.912002\\
198390.705993\\
205560.611999\\
205478.828994\\
203360.357002\\
199114.817001\\
201293.845001\\
207830.158004\\
197698.594001\\
203239.609001\\
197611.225997\\

  \end{tabular}
  \begin{tabular}{|c|}
    \hline
    Opção 3 [ms] \\
    \hline
    203455.275001\\
211206.041999\\
197908.959999\\
202196.333999\\
198711.988998\\
200473.372001\\
208630.888000\\
207026.652000\\
218691.422004\\
207706.132995\\
222586.570999\\
197973.020004\\
211164.223999\\
225928.515998\\
198874.804000\\
198840.096000\\
205587.041999\\
207217.750999\\
208344.699996\\
202186.023002\\
196298.675003\\
199305.034004\\
195543.421997\\
211049.842002\\
201241.436996\\
196029.867996\\
213466.261001\\
197755.699996\\
215781.550003\\
202633.296997\\
207692.336997\\
199085.402000\\
209131.826995\\
199962.319999\\
200131.684997\\
211305.824996\\
199024.714996\\
199656.143997\\
200330.039993\\
235347.397994\\
314829.455001\\
197507.366004\\
195592.687995\\
198117.076995\\
218314.411003\\
197601.516998\\
696983.492996\\
201467.684997\\
222263.551002\\
308753.834999\\
199199.266006\\
200401.573997\\
913252.808998\\
245864.714996\\
196003.160995\\
197564.656997\\
211258.578002\\
203892.424995\\
211127.557998\\
204189.597999\\
209724.394996\\
210252.112998\\
211643.968002\\
201642.031997\\
197560.337997\\
213216.848999\\
212245.299995\\
197395.237998\\
196868.642997\\
200111.273994\\
219745.275001\\
212442.816001\\
212264.862998\\
196730.083000\\
226795.617004\\
199681.481994\\
196140.077003\\
215272.972999\\
213744.094001\\
212239.170997\\
211371.820999\\
201663.841995\\
208548.171997\\
199326.139999\\
208015.150001\\
197828.065002\\
226185.018997\\
226055.114997\\
199736.810997\\
202952.624000\\
198478.914001\\
199884.455001\\
198851.379997\\
198868.838996\\
218355.901000\\
197110.048004\\
195512.235000\\
224791.856994\\
200772.836997\\
199104.977996\\

  \end{tabular}
  \begin{tabular}{|c|}
    \hline
    Opção 4 [ms] \\
    \hline
    387384.000000\\
374474.000000\\
427843.000000\\
382172.000000\\
378222.000000\\
376055.000000\\
376315.000000\\
194417.000000\\
380161.000000\\
379678.000000\\
405222.000000\\
388630.000000\\
378038.000000\\
372268.000000\\
382632.000000\\
197574.000000\\
580704.000000\\
380162.000000\\
404885.000000\\
385934.000000\\
377787.000000\\
381415.000000\\
648557.000000\\
374264.000000\\
388695.000000\\
384329.000000\\
207618.000000\\
379876.000000\\
376061.000000\\
213692.000000\\
376815.000000\\
226705.000000\\
219818.000000\\
557030.000000\\
373069.000000\\
382635.000000\\
382833.000000\\
384977.000000\\
383732.000000\\
391522.000000\\
207540.000000\\
574534.000000\\
375237.000000\\
195837.000000\\
600420.000000\\
377294.000000\\
195515.000000\\
378985.000000\\
379188.000000\\
208805.000000\\
570363.000000\\
372253.000000\\
211366.000000\\
388508.000000\\
392704.000000\\
495126.000000\\
407102.000000\\
211494.000000\\
384187.000000\\
382093.000000\\
211832.000000\\
372351.000000\\
375555.000000\\
572923.000000\\
386961.000000\\
386677.000000\\
392726.000000\\
380430.000000\\
384439.000000\\
604662.000000\\
389692.000000\\
398815.000000\\
562676.000000\\
372192.000000\\
375258.000000\\
196175.000000\\
418567.000000\\
372079.000000\\
194860.000000\\
422544.000000\\
377544.000000\\
374463.000000\\
376970.000000\\
376326.000000\\
400862.000000\\
391890.000000\\
206752.000000\\
412661.000000\\
390881.000000\\
223366.000000\\
373351.000000\\
400206.000000\\
575655.000000\\
388674.000000\\
383672.000000\\
198639.000000\\
377000.000000\\
372877.000000\\
387913.000000\\
435362.000000\\

  \end{tabular}
  \begin{tabular}{|c|}
    \hline
    Opção 5 [ms] \\
    \hline
    798799.000000\\
809658.000000\\
806619.000000\\
807654.000000\\
841950.000000\\
804294.000000\\
794001.000000\\
803674.000000\\
823421.000000\\
836823.000000\\
812663.000000\\
827006.000000\\
802566.000000\\
200806.000000\\
791945.000000\\
810600.000000\\
829574.000000\\
819217.000000\\
806333.000000\\
790672.000000\\
815474.000000\\
860111.000000\\
796783.000000\\
875436.000000\\
903562.000000\\
799664.000000\\
805160.000000\\
877402.000000\\
802668.000000\\
793100.000000\\
819994.000000\\
790226.000000\\
789644.000000\\
803754.000000\\
794327.000000\\
793979.000000\\
810330.000000\\
841225.000000\\
818517.000000\\
799198.000000\\
789596.000000\\
787954.000000\\
821004.000000\\
799028.000000\\
785814.000000\\
1103251.000000\\
808306.000000\\
798444.000000\\
840483.000000\\
805219.000000\\
812946.000000\\
1204824.000000\\
803446.000000\\
785016.000000\\
801234.000000\\
810304.000000\\
787603.000000\\
816566.000000\\
803445.000000\\
822669.000000\\
789270.000000\\
788115.000000\\
785006.000000\\
808844.000000\\
796992.000000\\
824105.000000\\
788755.000000\\
794453.000000\\
811171.000000\\
799804.000000\\
799161.000000\\
787083.000000\\
789392.000000\\
795363.000000\\
788889.000000\\
801048.000000\\
848622.000000\\
820199.000000\\
796829.000000\\
802407.000000\\
813252.000000\\
795081.000000\\
818911.000000\\
829892.000000\\
803175.000000\\
801467.000000\\
785462.000000\\
820895.000000\\
880082.000000\\
791788.000000\\
788466.000000\\
817210.000000\\
828234.000000\\
788479.000000\\
791639.000000\\
836845.000000\\
786796.000000\\
802906.000000\\
788881.000000\\
801990.000000\\

  \end{tabular}
  \begin{tabular}{|c|}
    \hline
    Opção 6 [ms] \\
    \hline
    206021.424995\\
221135.761001\\
194634.906005\\
208775.874000\\
227648.604003\\
201682.800003\\
209009.982002\\
196474.446006\\
210193.853996\\
210716.194000\\
196513.631004\\
198629.799995\\
208428.291000\\
212321.134994\\
200682.558006\\
201131.635002\\
222372.802001\\
199312.712005\\
196317.111999\\
196014.212997\\
201738.277999\\
198154.059997\\
207953.881996\\
197331.064002\\
198204.980003\\
197909.973999\\
194491.701995\\
211310.005996\\
196021.000999\\
211372.725997\\
228070.555999\\
195635.835998\\
197955.312004\\
196819.007003\\
194700.744995\\
196085.821998\\
202845.762001\\
209316.425003\\
230620.589004\\
201132.019004\\
225474.807998\\
195188.798995\\
198957.187004\\
205883.418998\\
194414.484001\\
238461.251998\\
206435.582000\\
199757.081001\\
213866.972000\\
225913.565002\\
196546.166000\\
213463.575996\\
195959.438003\\
206839.480995\\
220540.803001\\
209264.978004\\
197839.481994\\
392216.760002\\
199549.319000\\
197799.188003\\
203430.643997\\
208214.865997\\
213407.930999\\
202921.930000\\
215385.958999\\
220633.361999\\
197105.193000\\
197583.671005\\
192261.052001\\
194151.289001\\
215815.277999\\
194891.460998\\
201080.104003\\
198913.440002\\
202933.259002\\
198982.377998\\
195121.241996\\
202507.513999\\
197895.453002\\
199605.223999\\
210042.898002\\
197364.318000\\
198122.929000\\
195653.922996\\
210346.429000\\
215784.079002\\
210709.113998\\
288172.521995\\
207316.75\\
204968.500999\\
196863.434997\\
200143.079002\\
217198.010002\\
208987.342994\\
207090.613998\\
194524.899002\\
199193.183006\\
221173.065002\\
195255.782997\\
195518.802001\\

  \end{tabular}
  \caption{Tabela de tempo total}{|c|}
\end{table}
\end{document}
