%*************************************************
\documentclass[a4paper,10pt]{article}
\usepackage[brazil]{babel}
\usepackage[utf8]{inputenc}
%codigo
%\usepackage[latin1]{inputenc}
\usepackage{amsthm,amsfonts,amsmath,amssymb}
%img
\usepackage{graphicx}
\usepackage{subfig}
%tabela

%fim
%fim
\usepackage{listings} 
\usepackage{makeidx}
\usepackage{enumerate}
\usepackage{hyperref}
\hypersetup{
  colorlinks,
  linkcolor=blue,
  filecolor=blue,
  urlcolor=blue,
  citecolor=blue 
}

%titulo
\title{Servidor RMI}
\author{Leandro Kümmel Tria Mendes RA033910 \\ Fernando Teixeira RA085858}
\makeindex
%inicio
\begin{document}
\maketitle
\begin{figure}[!htb]
  \centering
  \includegraphics[scale=0.5]{logo.jpg}
\end{figure}
\newpage
\tableofcontents
\listoffigures
\listoftables
\newpage
\section{Introdução}
O objetivo desse projeto é implementar um sistema cliente/servidor, com operações para o gerenciamento de livros em uma livraria.
Na comunicação entre cliente e servidor utilizou-se a interface de programação de chamadas remotas RMI\footnote{
Mais informações sobre RMI \url{http://docs.oracle.com/javase/tutorial/rmi/overview.html} } e a partir da execução de testes podemos avaliar alguns aspectos
desse protocolo e posteriormente compará-lo com os projetos anteriores, no qual foi utilizado o TCP como protocolo.
\section{Desenvolvimento}
Linux foi o sistema operacional utilizado para o desenvolvimento (distribuição
2.6.43.8-1.fc15.i686). Igualmente, para os testes utilizou-se duas máquinas com linux, porém com distribuições diferentes.
\subsection{RMI - Remote Method Invocation}
O nível de abstração do RMI, para o programador, é maior em comparação aos projetos anteriores. Esse sistema possui um conjunto de operações e recursos que 
estão envolvidos no processo de invocação do método remoto. A implementação do RMI é essencialmente feita em três camadas, que por ordem de \"proximidade do programador\", são: 
\begin{itemize}
\item \label{itm:stubskeleton}\emph{Camada Stub/Skeleton}: Podemos classificar o Stub, que reside no lado do cliente, como um proxie para um objeto remoto, ou seja, o Stub é responsável por receber
os parâmetros dos métodos e encaminha-los para o Skeleton.  Esse no entando, residente do servidor, recebe os parâmetros enviados pelo primeiro e executa as chamadas
no objeto remoto. O Skeleton também é responsável por entregar ao Stub o retorno dos métodos remoto, e esse por fim, lê os resultados e retorna os valores ao 
objeto que inicialmente executou a chamada.
\item \emph{Camada de referência remota - RRL}: Logo abaixo da camada Stub/Skeleton[\ref{itm:stubskeleton}] a RRL gerência e interpreta as refências remotas. Essa camada estabelece a semântica da ligação RMI, e as refêrencias entre o cliente e servidor (objetos remotos) é unicast, em outras palavras, um Stub para um Skeleton.
\item \label{itm:tranportetcp} \emph{Camada de transporte TCP}: Essa última camada do RMI assegura a ligação entre as VM\footnote{Virtual Machine: http://en.wikipedia.org/wiki/VirtualMachine} através do protocolo TCP/IP, já especificado no projeto anterior. 
\end{itemize}
\subsection{Implementação}
O projeto conta apenas com quatro diretórios, bin, src, relatorio e estat e um README.md\footnote{Leia esse arquivo antes de
executar o sistema} para instruções adicionais. Há também um Makefile, por diretório e subdiretório, o qual 
compila e executa o sistema.
\\Os diretório são:
\begin{enumerate}[I.]
  \label{sec:dirs}
\item \label{itm:bin} \emph{bin}: Contém os binários do projeto, incluindo o stub/skeletons[\ref{itm:stubskeleton}].
\item \label{itm:src} \emph{src}: Contém os pacotes, e classes, do projeto. Importante lembrar que há o mysql-connector.jar, necessário
  para a conexão com o banco de dados para buscar resultados da biblioteca.
\item \label{itm:estat} \emph{estat}: Medidas de tempo efetuadas pelo teste.
\item \label{itm:relatorio} \emph{relatorio}
\end{enumerate}

O diretório src[\ref{itm:src}] contém os pacotes do projeto, que são:
\begin{enumerate}[I.] 
  \label{sec:packs}
\item \label{itm:common} \emph{common}: Contém algumas variaveis/classes comum tanto ao cliente, quanto ao servidor. Há também a classe que marca e calcula o tempo
  e as estatísticas.
\item \label{itm:biblioteca} \emph{biblioteca}: Contém as classes que manipulam a biblioteca, incluindo os métodos distribuidos, em outras palavras, os métodos que são
  acessados pelo cliente, ou seja, esse pacote pertence ao servidor.
\item \label{itm:client} \emph{client}: Pacote exclusivo do cliente, que contém apenas as classes de execução do cliente e dos testes.
\item \label{itm:server} \emph{server}: Pacote exclusivo do servidor, que contém apenas a classes responsável por controle do servidor, ou seja,
  faz o \"rebind\" do mesmo esperando por conexões dos clientes.
\end{enumerate}

\subsubsection{Manipulação de dados}
Diferentemente dos projetos anteriores, a manipulação dos dados é feita em conjunto com um banco de dados ( MySQL\footnote{\url{http://www.mysql.com/}} ) portanto,
ganha-se em simplicidade de implementação.
\\Arquivos presentes nos pacotes \emph{common}[\ref{itm:common}] e \emph{biblioteca}[\ref{itm:biblioteca}] :
\\\textbf{Common}:
\begin{enumerate}[I.]
\item \label{itm:commonvarsj}\emph{CommonVars.java}: Contém as variáveis estáticas utilizadas por ambos os lados do projetos, ou seja, tanto pelo cliente quanto pelo servidor.
\item \label{itm:inoutj} \emph{InOut.java}: Controla a parte de entrada/saída do programa, assim como a escrita, das estatísticas, em arquivos.
\item \label{itm:tempoj} \emph{Tempo.java}: Classe que contém os métodos que controla o tempo em nanosegundos (esse futuramente é transformado em milisegundos,
  para carater de comparação.
\end{enumerate}
\textbf{Bibliotecas}:
\begin{enumerate}[I.]
\item \label{itm:bibliotecaj} \emph{Biblioteca.java}: Interface para a BibliotecaImpl, citada abaixo.
\item \label{itm:bibliotecaimplj} \emph{BibliotecaImpl.java}: Implementa a interface Biblioteca e contém os métodos acessados pelo cliente. Há também a conexão e consulta junto
  ao banco de dados.
\item \label{itm:livroj} \emph{Livro.java}: Objeto Livro, contém os atributos do mesmo, tal como, isbn, quantidade, título.
\item \label{itm:menuj} \emph{Menu.java}: Gerencia o menu básico do sistema.
\item \label{itm:usuarioj} \emph{Usuario.java}: Gerencia o login necessário para editar a quantidade de um livro.
\end{enumerate}
\subsubsection{Conexão Servidor/Cliente}
Compreende dois pacotes \emph{server}[\ref{itm:server}] e \emph{client}[\ref{itm:client}]
\\\textbf{Servidor}:
\begin{enumerate}[I.]
\item \label{itm:serverj} \emph{Server.java}: Responsável por fazer o \"rebind\" do servidor, ou seja, nomea-lo e abrir a porta especificada em CommonVars.java[\ref{itm:commonvarsj}], além de atribuir o 
skeleton. A inicialização do RMI é feito durante o tempo de execução.
\end{enumerate}
\textbf{Cliente}:
\begin{enumerate}[I.]
\item \label{itm:clientj} \emph{Cliente.java}: Faz o \"lookup\" pelo servidor, ou seja, pesquisa pelo endereço do servidor, dado em CommonVars.java[\ref{itm:commonvarsj}].
\item \label{itm:testerj} \emph{Tester.java}: Mesma estrutura que o Cliente.java[\ref{itm:clientj}], porém executa os testes comparativos com os projetos anteriores.
\end{enumerate}
\subsection{Coleta e gerência de dados para testes}
Para realizar os testes implementou-se alguns arquivos adicionas[\ref{itm:tempoj}] [\ref{itm:testerj}]. Uma constante, denominada \emph{TOTAL}\footnote{Nesse sistema consideramos TOTAL igual a 100} , contém o número de testes a serem realizados, ou seja, cada opção do \emph{menu}[\ref{itm:menuj}] é executada \emph{TOTAL} vezes. Todos os dados são 
salvos no diretório \emph{estat}[\ref{itm:estat}].
\subsection{Vantagens da implementação}
O sistema de livraria é um sistema robusto e com baixa complexidade de tempo.
A escolha do banco de dados mysql, possibilitou em boa 
performance em questão de tempo de processamento e também um maior nível de abstração, uma vez que, essa estrutura
mostrou-se eficaz para o problema e tem melhor complexidade de tempo com 
relação a outras estruturas, como a simples leitura em arquivos, além de ser da implementação e manuntenção serem
simples.
\\ Com relação a comunicação entre cliente/servidor, o nível de abstração é maior em comparação com os projetos anteriores, por exemplo, a não preocupação com a 
leitura do buffer, como havia no caso do TCP e UDP, também não é necessário se preocupar com os \"zumbis\" presentes no protocolo TCP.
\newpage
\begin{figure}[!htb]
  \centering
  \subfloat[Fluxograma Servidor]{
  \includegraphics[width=150px,height=300px]{fluxo_servidor.png}
  \label{fluxoservidor}
  }
  \quad %espaco separador
   \subfloat[Fluxograma Cliente]{
     \label{fluxoserver}
     \includegraphics[width=150px,height=300px]{fluxo_cliente.png}
   }
   \caption{Fluxogramas}
   \label{figfluxos}
\end{figure}
\newpage
\begin{figure}[!htb]
  \centering
  \label{fluxotempo}
  \includegraphics[scale=0.5]{fluxo_tempo.png}
  \caption{Definição do cálculo dos tempos}
\end{figure}
\newpage
\section{Resultados e discussões}
Os testes foram efetuados em duas máquinas, ambas conectadas à rede porém, não localmente. Denominaremos a máquina servidor como [S] e a cliente como [C]. [S] e [C] estão em continentes diferentes.
\\Foram efetuadas 100 medições para cada opção do \emph{menu}[\ref{itm:menuj}], ao todo foram 600 medições de tempo. Dividiu-se o tempo em tempo de processamento e tempo de comunicação, sendo o último a diferença entre o tempo total e o tempo de processamento.
\\Abaixo demostramos o \emph{traceroute}\footnote{\url{http://en.wikipedia.org/wiki/Traceroute}}, o qual mostra a rota percorrida por um pequeno pacote, marcando assim os roteadores por qual passa. Vale ressaltar que os teste 
foram realizados em uma rede sem fio, sendo assim há uma perda maior de pacotes, e quanto maior a rota a ser percorrida, também espera-se uma perda maior, porém 
como descrito anteriormente[\ref{itm:transportetcp}], o RMI utilizada na camada de transporte o protocolo TCP, o qual garante a entrega, em ordem, da mensagem.
\lstinputlisting{traceroute}
\subsection{Tabelas e gráficos}
\subsubsection{Tempo total}[\ref{fluxotempo}]
Representaremos todas as 100 medidas das 6 opções do \emph{menu}[\ref{itm:menuj}]
\begin{table}
  \tiny
  \centering
  \begin{tabular}{|c|}
    \hline
    Opção 1 [ms] \\
    \hline
    186134.000000\\
187389.000000\\
187232.000000\\
186872.000000\\
184125.000000\\
186449.000000\\
186252.000000\\
189478.000000\\
183976.000000\\
208636.000000\\
201516.000000\\
208826.000000\\
199975.000000\\
187522.000000\\
187232.000000\\
186570.000000\\
184801.000000\\
187593.000000\\
187224.000000\\
199953.000000\\
184321.000000\\
184627.000000\\
185264.000000\\
182234.000000\\
185997.000000\\
183630.000000\\
195198.000000\\
185220.000000\\
184872.000000\\
184440.000000\\
215691.000000\\
185690.000000\\
185212.000000\\
184893.000000\\
184262.000000\\
187784.000000\\
190727.000000\\
186847.000000\\
187853.000000\\
187898.000000\\
184341.000000\\
192566.000000\\
192088.000000\\
195817.000000\\
189500.000000\\
187443.000000\\
197499.000000\\
188974.000000\\
187302.000000\\
187996.000000\\
207983.000000\\
183661.000000\\
188076.000000\\
193033.000000\\
203251.000000\\
191111.000000\\
208720.000000\\
191228.000000\\
208735.000000\\
192248.000000\\
188302.000000\\
199366.000000\\
185200.000000\\
189354.000000\\
184946.000000\\
212453.000000\\
186990.000000\\
187158.000000\\
186925.000000\\
187869.000000\\
190985.000000\\
192168.000000\\
188490.000000\\
198904.000000\\
211250.000000\\
190856.000000\\
186611.000000\\
184224.000000\\
188581.000000\\
200660.000000\\
187730.000000\\
183613.000000\\
182749.000000\\
184443.000000\\
191332.000000\\
186974.000000\\
183429.000000\\
188809.000000\\
187492.000000\\
189442.000000\\
188319.000000\\
212336.000000\\
184941.000000\\
199550.000000\\
188052.000000\\
184290.000000\\
189488.000000\\
187303.000000\\
187554.000000\\
184799.000000\\
    
    \hline
  \end{tabular}
  \begin{tabular}{|c|}
    \hline
    Opção 2 [ms] \\
    \hline
    201859.0\\
202306.817001\\
199848.501998\\
201010.160995\\
199215.401000\\
208501.749000\\
201476.560997\\
198648.371002\\
198738.604995\\
198555.204002\\
202348.875\\
204144.968002\\
223860.324996\\
202700.039993\\
193195.859001\\
199188.389999\\
203041.983001\\
214173.309997\\
216054.957000\\
197666.222000\\
196235.888999\\
207876.913002\\
199879.262001\\
198508.363998\\
207005.726997\\
210798.0\\
210775.443000\\
203899.921005\\
204159.311996\\
200577.259994\\
195458.647994\\
221667.504005\\
209850.762001\\
198213.652999\\
211319.792999\\
209023.158004\\
210316.257003\\
216141.311004\\
205385.445999\\
209093.493995\\
225615.328002\\
198674.750999\\
195472.144004\\
205143.945999\\
208333.218002\\
195988.058998\\
198926.396003\\
202343.136001\\
203655.444000\\
209972.173004\\
200545.443000\\
197275.022003\\
200417.637001\\
219973.697998\\
200157.276000\\
220715.338005\\
207236.249000\\
196315.908996\\
197564.450996\\
200287.746002\\
212831.952003\\
307414.807998\\
240003.151000\\
208634.329002\\
200791.722999\\
216155.674003\\
201785.068000\\
195292.268997\\
195221.592002\\
200974.921005\\
212591.645996\\
195282.427001\\
198348.010993\\
197866.603004\\
213719.018005\\
202181.004997\\
199644.747001\\
197390.852996\\
201281.739997\\
202089.450004\\
211811.261001\\
196298.339996\\
220099.944000\\
197679.093002\\
211163.772003\\
213207.867996\\
219191.463996\\
211865.255996\\
200830.794006\\
210114.912002\\
198390.705993\\
205560.611999\\
205478.828994\\
203360.357002\\
199114.817001\\
201293.845001\\
207830.158004\\
197698.594001\\
203239.609001\\
197611.225997\\

    \hline
  \end{tabular}
  \begin{tabular}{|c|}
    \hline
    Opção 3 [ms] \\
    \hline
    203455.275001\\
211206.041999\\
197908.959999\\
202196.333999\\
198711.988998\\
200473.372001\\
208630.888000\\
207026.652000\\
218691.422004\\
207706.132995\\
222586.570999\\
197973.020004\\
211164.223999\\
225928.515998\\
198874.804000\\
198840.096000\\
205587.041999\\
207217.750999\\
208344.699996\\
202186.023002\\
196298.675003\\
199305.034004\\
195543.421997\\
211049.842002\\
201241.436996\\
196029.867996\\
213466.261001\\
197755.699996\\
215781.550003\\
202633.296997\\
207692.336997\\
199085.402000\\
209131.826995\\
199962.319999\\
200131.684997\\
211305.824996\\
199024.714996\\
199656.143997\\
200330.039993\\
235347.397994\\
314829.455001\\
197507.366004\\
195592.687995\\
198117.076995\\
218314.411003\\
197601.516998\\
696983.492996\\
201467.684997\\
222263.551002\\
308753.834999\\
199199.266006\\
200401.573997\\
913252.808998\\
245864.714996\\
196003.160995\\
197564.656997\\
211258.578002\\
203892.424995\\
211127.557998\\
204189.597999\\
209724.394996\\
210252.112998\\
211643.968002\\
201642.031997\\
197560.337997\\
213216.848999\\
212245.299995\\
197395.237998\\
196868.642997\\
200111.273994\\
219745.275001\\
212442.816001\\
212264.862998\\
196730.083000\\
226795.617004\\
199681.481994\\
196140.077003\\
215272.972999\\
213744.094001\\
212239.170997\\
211371.820999\\
201663.841995\\
208548.171997\\
199326.139999\\
208015.150001\\
197828.065002\\
226185.018997\\
226055.114997\\
199736.810997\\
202952.624000\\
198478.914001\\
199884.455001\\
198851.379997\\
198868.838996\\
218355.901000\\
197110.048004\\
195512.235000\\
224791.856994\\
200772.836997\\
199104.977996\\

    \hline
  \end{tabular}
  \begin{tabular}{|c|}
    \hline
    Opção 4 [ms] \\
    \hline
    387384.000000\\
374474.000000\\
427843.000000\\
382172.000000\\
378222.000000\\
376055.000000\\
376315.000000\\
194417.000000\\
380161.000000\\
379678.000000\\
405222.000000\\
388630.000000\\
378038.000000\\
372268.000000\\
382632.000000\\
197574.000000\\
580704.000000\\
380162.000000\\
404885.000000\\
385934.000000\\
377787.000000\\
381415.000000\\
648557.000000\\
374264.000000\\
388695.000000\\
384329.000000\\
207618.000000\\
379876.000000\\
376061.000000\\
213692.000000\\
376815.000000\\
226705.000000\\
219818.000000\\
557030.000000\\
373069.000000\\
382635.000000\\
382833.000000\\
384977.000000\\
383732.000000\\
391522.000000\\
207540.000000\\
574534.000000\\
375237.000000\\
195837.000000\\
600420.000000\\
377294.000000\\
195515.000000\\
378985.000000\\
379188.000000\\
208805.000000\\
570363.000000\\
372253.000000\\
211366.000000\\
388508.000000\\
392704.000000\\
495126.000000\\
407102.000000\\
211494.000000\\
384187.000000\\
382093.000000\\
211832.000000\\
372351.000000\\
375555.000000\\
572923.000000\\
386961.000000\\
386677.000000\\
392726.000000\\
380430.000000\\
384439.000000\\
604662.000000\\
389692.000000\\
398815.000000\\
562676.000000\\
372192.000000\\
375258.000000\\
196175.000000\\
418567.000000\\
372079.000000\\
194860.000000\\
422544.000000\\
377544.000000\\
374463.000000\\
376970.000000\\
376326.000000\\
400862.000000\\
391890.000000\\
206752.000000\\
412661.000000\\
390881.000000\\
223366.000000\\
373351.000000\\
400206.000000\\
575655.000000\\
388674.000000\\
383672.000000\\
198639.000000\\
377000.000000\\
372877.000000\\
387913.000000\\
435362.000000\\

    \hline
  \end{tabular}
  \begin{tabular}{|c|}
    \hline
    Opção 5 [ms] \\
    \hline
    798799.000000\\
809658.000000\\
806619.000000\\
807654.000000\\
841950.000000\\
804294.000000\\
794001.000000\\
803674.000000\\
823421.000000\\
836823.000000\\
812663.000000\\
827006.000000\\
802566.000000\\
200806.000000\\
791945.000000\\
810600.000000\\
829574.000000\\
819217.000000\\
806333.000000\\
790672.000000\\
815474.000000\\
860111.000000\\
796783.000000\\
875436.000000\\
903562.000000\\
799664.000000\\
805160.000000\\
877402.000000\\
802668.000000\\
793100.000000\\
819994.000000\\
790226.000000\\
789644.000000\\
803754.000000\\
794327.000000\\
793979.000000\\
810330.000000\\
841225.000000\\
818517.000000\\
799198.000000\\
789596.000000\\
787954.000000\\
821004.000000\\
799028.000000\\
785814.000000\\
1103251.000000\\
808306.000000\\
798444.000000\\
840483.000000\\
805219.000000\\
812946.000000\\
1204824.000000\\
803446.000000\\
785016.000000\\
801234.000000\\
810304.000000\\
787603.000000\\
816566.000000\\
803445.000000\\
822669.000000\\
789270.000000\\
788115.000000\\
785006.000000\\
808844.000000\\
796992.000000\\
824105.000000\\
788755.000000\\
794453.000000\\
811171.000000\\
799804.000000\\
799161.000000\\
787083.000000\\
789392.000000\\
795363.000000\\
788889.000000\\
801048.000000\\
848622.000000\\
820199.000000\\
796829.000000\\
802407.000000\\
813252.000000\\
795081.000000\\
818911.000000\\
829892.000000\\
803175.000000\\
801467.000000\\
785462.000000\\
820895.000000\\
880082.000000\\
791788.000000\\
788466.000000\\
817210.000000\\
828234.000000\\
788479.000000\\
791639.000000\\
836845.000000\\
786796.000000\\
802906.000000\\
788881.000000\\
801990.000000\\

    \hline
  \end{tabular}
  \begin{tabular}{|c|}
    \hline
    Opção 6 [ms] \\
    \hline
    206021.424995\\
221135.761001\\
194634.906005\\
208775.874000\\
227648.604003\\
201682.800003\\
209009.982002\\
196474.446006\\
210193.853996\\
210716.194000\\
196513.631004\\
198629.799995\\
208428.291000\\
212321.134994\\
200682.558006\\
201131.635002\\
222372.802001\\
199312.712005\\
196317.111999\\
196014.212997\\
201738.277999\\
198154.059997\\
207953.881996\\
197331.064002\\
198204.980003\\
197909.973999\\
194491.701995\\
211310.005996\\
196021.000999\\
211372.725997\\
228070.555999\\
195635.835998\\
197955.312004\\
196819.007003\\
194700.744995\\
196085.821998\\
202845.762001\\
209316.425003\\
230620.589004\\
201132.019004\\
225474.807998\\
195188.798995\\
198957.187004\\
205883.418998\\
194414.484001\\
238461.251998\\
206435.582000\\
199757.081001\\
213866.972000\\
225913.565002\\
196546.166000\\
213463.575996\\
195959.438003\\
206839.480995\\
220540.803001\\
209264.978004\\
197839.481994\\
392216.760002\\
199549.319000\\
197799.188003\\
203430.643997\\
208214.865997\\
213407.930999\\
202921.930000\\
215385.958999\\
220633.361999\\
197105.193000\\
197583.671005\\
192261.052001\\
194151.289001\\
215815.277999\\
194891.460998\\
201080.104003\\
198913.440002\\
202933.259002\\
198982.377998\\
195121.241996\\
202507.513999\\
197895.453002\\
199605.223999\\
210042.898002\\
197364.318000\\
198122.929000\\
195653.922996\\
210346.429000\\
215784.079002\\
210709.113998\\
288172.521995\\
207316.75\\
204968.500999\\
196863.434997\\
200143.079002\\
217198.010002\\
208987.342994\\
207090.613998\\
194524.899002\\
199193.183006\\
221173.065002\\
195255.782997\\
195518.802001\\

    \hline
  \end{tabular}
  \caption{Tabela de tempo total}
\end{table}
\newpage
\subsubsection{Tempo de processamento}[\ref{fluxotempo}]
Representaremos todas as 100 medidas das 6 opções do \emph{menu}[\ref{itm:menuj}]
\begin{table}
  \tiny
  \centering
  \begin{tabular}{|c|}
    \hline
    Opção 1 [ms] \\
    \hline
    23499.42675\\
2741.711914\\
2426.067382\\
2293.096679\\
2483.584960\\
2732.129882\\
2496.462890\\
2700.336914\\
2766.775390\\
2358.172851\\
2197.652343\\
2110.012695\\
1998.833007\\
2033.600585\\
15533.80859\\
3161.451171\\
2274.647460\\
2100.997070\\
2092.358398\\
2376.628906\\
2383.828125\\
2089.916992\\
2098.202148\\
2298.115234\\
2955.897460\\
2044.863281\\
1960.921875\\
2085.931640\\
1999.516601\\
1985.002929\\
2314.264648\\
3211.170898\\
2230.250976\\
2409.091796\\
2142.537109\\
2088.868164\\
2257.673828\\
2641.204101\\
2255.297851\\
1825.700195\\
2079.362304\\
3437.389648\\
1926.607421\\
1862.973632\\
1717.3359375\\
2198.416015\\
2468.734375\\
2300.391601\\
2601.793945\\
1718.525390\\
1825.108398\\
1918.760742\\
1763.375\\
1587.811523\\
4070.6875\\
1531.560546\\
1527.393554\\
1513.079101\\
1525.045898\\
1734.748046\\
1660.745117\\
1561.499023\\
1903.579101\\
1511.046875\\
2468.661132\\
1588.126953\\
1541.253906\\
1734.319335\\
1748.565429\\
1281.558593\\
1318.267578\\
1490.784179\\
1541.271484\\
1308.565429\\
1313.391601\\
1450.140625\\
1223.756835\\
1944.65625\\
1305.014648\\
1267.692382\\
1641.869140\\
2623.924804\\
1224.431640\\
1249.188476\\
1685.751953\\
1371.079101\\
1380.0625\\
2513.627929\\
1259.575195\\
1453.826171\\
1326.1875\\
1325.004882\\
2211.475585\\
1201.508789\\
1698.415039\\
1963.515625\\
1187.314453\\
1281.374023\\
1325.030273\\
1306.115234\\

    \hline
  \end{tabular}
  \begin{tabular}{|c|}
    \hline
    Opção 2 [ms] \\
    \hline
    2588.69042\\
2166.27148\\
2984.88476\\
2306.31054\\
2287.46679\\
3056.28710\\
2742.53710\\
2058.77929\\
2033.60839\\
2092.90527\\
2037.83398\\
2029.88183\\
2053.88671\\
1992.09375\\
1939.20410\\
1906.79589\\
1929.55371\\
1996.64062\\
2156.50292\\
1977.66015\\
2045.03320\\
2102.73828\\
2213.47363\\
1944.46582\\
2150.33398\\
1784.53515\\
2122.51562\\
2024.09765\\
1937.89843\\
2206.93847\\
2116.26074\\
2156.47656\\
1826.13378\\
1969.08007\\
1745.32519\\
1863.08691\\
1610.82519\\
2063.41601\\
2386.09082\\
1649.90136\\
2705.86132\\
1729.25097\\
1865.64257\\
1663.25292\\
1903.49804\\
2441.68457\\
1757.96777\\
1627.75585\\
1736.50781\\
2048.62695\\
2008.40332\\
1573.51953\\
2080.39160\\
1725.84277\\
1640.64648\\
1443.74121\\
1933.15820\\
2116.26660\\
1625.32910\\
1714.30273\\
1802.28710\\
1583.49414\\
1719.99902\\
1610.22070\\
1849.58886\\
1530.31738\\
2223.86230\\
1616.85156\\
1421.30468\\
2223.39160\\
1423.40820\\
1587.52441\\
3861.52050\\
1734.07714\\
1808.31738\\
1615.26269\\
2609.58300\\
1638.81835\\
1663.13281\\
1699.91308\\
2133.82617\\
1397.0\\
1598.37109\\
1326.52441\\
1423.82812\\
1680.19140\\
1429.39355\\
1819.98437\\
1721.93847\\
1437.08105\\
1474.50097\\
1431.42285\\
1599.43457\\
1927.44531\\
1943.12304\\
1764.84277\\
1490.30273\\
1784.49218\\
1686.62794\\
1413.28906\\

    \hline
  \end{tabular}
  \begin{tabular}{|c|}
    \hline
    Opção 3 [ms] \\
    \hline
    3674.17187\\
2809.39257\\
2807.36425\\
2515.47070\\
2921.58496\\
3067.58593\\
2541.833984\\
2312.737304\\
2513.616210\\
2345.302734\\
2515.03125\\
2288.631835\\
2425.683593\\
2267.762695\\
2315.449218\\
2256.185546\\
2201.988281\\
2189.830078\\
2962.610351\\
2055.752929\\
3069.470703\\
2143.847656\\
2408.840820\\
2326.465820\\
2477.500976\\
2418.101562\\
4408.851562\\
2209.492187\\
2048.783203\\
3774.604492\\
2004.117187\\
2419.379882\\
2079.896484\\
6134.481445\\
1894.795898\\
2232.594726\\
2940.666992\\
1885.816403\\
1824.250976\\
2054.471679\\
1824.631835\\
2103.592773\\
2128.505859\\
1940.451171\\
2138.087890\\
1994.018554\\
2152.935546\\
2364.742187\\
4888.248046\\
2240.315429\\
2052.795898\\
3819.780273\\
2134.796875\\
1930.450195\\
1781.019531\\
1852.238281\\
1853.941406\\
1934.556640\\
1688.001953\\
4222.914062\\
2009.717773\\
2002.026367\\
1904.137695\\
1703.552734\\
1850.693359\\
1758.201171\\
1747.850585\\
1685.582031\\
2361.884765\\
1656.470703\\
1749.207031\\
1746.084960\\
2101.828125\\
1789.326171\\
2061.145507\\
1625.257812\\
1787.131835\\
1682.167968\\
1763.418945\\
1989.556640\\
2044.693359\\
2691.419921\\
1602.276367\\
1600.141601\\
1648.862304\\
1390.353515\\
1708.804687\\
1545.240234\\
2186.770507\\
1739.717773\\
1959.007812\\
1788.939453\\
1715.9375\\
2120.757812\\
1583.235351\\
1576.397460\\
1949.586914\\
1911.683593\\
1786.504882\\
2769.505859\\

    \hline
  \end{tabular}
  \begin{tabular}{|c|}
    \hline
    Opção 4 [ms] \\
    \hline
    555.000000\\
545.000000\\
554.000000\\
577.000000\\
550.000000\\
547.000000\\
543.000000\\
550.000000\\
569.000000\\
545.000000\\
560.000000\\
569.000000\\
552.000000\\
531.000000\\
543.000000\\
541.000000\\
536.000000\\
561.000000\\
545.000000\\
543.000000\\
550.000000\\
554.000000\\
556.000000\\
551.000000\\
563.000000\\
581.000000\\
586.000000\\
565.000000\\
548.000000\\
551.000000\\
575.000000\\
552.000000\\
557.000000\\
547.000000\\
553.000000\\
556.000000\\
555.000000\\
543.000000\\
591.000000\\
582.000000\\
546.000000\\
545.000000\\
581.000000\\
550.000000\\
556.000000\\
596.000000\\
542.000000\\
551.000000\\
591.000000\\
549.000000\\
549.000000\\
547.000000\\
565.000000\\
567.000000\\
570.000000\\
622.000000\\
547.000000\\
553.000000\\
587.000000\\
570.000000\\
555.000000\\
553.000000\\
542.000000\\
551.000000\\
565.000000\\
551.000000\\
559.000000\\
560.000000\\
570.000000\\
547.000000\\
561.000000\\
557.000000\\
561.000000\\
558.000000\\
540.000000\\
566.000000\\
531.000000\\
545.000000\\
576.000000\\
562.000000\\
552.000000\\
560.000000\\
589.000000\\
548.000000\\
546.000000\\
546.000000\\
554.000000\\
573.000000\\
550.000000\\
552.000000\\
555.000000\\
566.000000\\
549.000000\\
558.000000\\
557.000000\\
600.000000\\
553.000000\\
545.000000\\
546.000000\\
547.000000\\

    \hline
  \end{tabular}
  \begin{tabular}{|c|}
    \hline
    Opção 5 [ms] \\
    \hline
    587293.000000\\
564101.000000\\
568421.000000\\
555928.000000\\
572451.000000\\
556187.000000\\
564367.000000\\
564026.000000\\
600243.000000\\
564148.000000\\
560172.000000\\
560065.000000\\
588025.000000\\
568018.000000\\
560106.000000\\
582428.000000\\
591891.000000\\
571918.000000\\
559977.000000\\
568044.000000\\
576041.000000\\
560274.000000\\
560677.000000\\
556033.000000\\
572326.000000\\
560592.000000\\
564051.000000\\
587935.000000\\
558832.000000\\
564474.000000\\
563958.000000\\
604257.000000\\
580073.000000\\
596047.000000\\
584255.000000\\
592302.000000\\
588301.000000\\
604191.000000\\
576113.000000\\
564300.000000\\
604714.000000\\
569860.000000\\
555921.000000\\
572221.000000\\
559985.000000\\
575807.000000\\
572493.000000\\
559996.000000\\
576394.000000\\
604193.000000\\
572449.000000\\
559829.000000\\
567938.000000\\
556361.000000\\
573747.000000\\
620089.000000\\
579521.000000\\
585557.000000\\
559933.000000\\
562390.000000\\
560382.000000\\
564387.000000\\
592450.000000\\
583946.000000\\
604141.000000\\
569690.000000\\
580572.000000\\
576014.000000\\
560146.000000\\
562996.000000\\
564238.000000\\
560074.000000\\
587962.000000\\
556263.000000\\
588097.000000\\
564699.000000\\
553780.000000\\
556397.000000\\
563761.000000\\
592242.000000\\
568260.000000\\
564199.000000\\
594113.000000\\
568013.000000\\
573846.000000\\
580129.000000\\
564235.000000\\
555879.000000\\
552448.000000\\
564098.000000\\
553049.000000\\
569645.000000\\
556187.000000\\
564195.000000\\
568313.000000\\
556060.000000\\
571894.000000\\
561857.000000\\
572339.000000\\
569532.000000\\

    \hline
  \end{tabular}
  \begin{tabular}{|c|}
    \hline
    Opção 6 [ms] \\
    \hline
    206021.424995\\
221135.761001\\
194634.906005\\
208775.874000\\
227648.604003\\
201682.800003\\
209009.982002\\
196474.446006\\
210193.853996\\
210716.194000\\
196513.631004\\
198629.799995\\
208428.291000\\
212321.134994\\
200682.558006\\
201131.635002\\
222372.802001\\
199312.712005\\
196317.111999\\
196014.212997\\
201738.277999\\
198154.059997\\
207953.881996\\
197331.064002\\
198204.980003\\
197909.973999\\
194491.701995\\
211310.005996\\
196021.000999\\
211372.725997\\
228070.555999\\
195635.835998\\
197955.312004\\
196819.007003\\
194700.744995\\
196085.821998\\
202845.762001\\
209316.425003\\
230620.589004\\
201132.019004\\
225474.807998\\
195188.798995\\
198957.187004\\
205883.418998\\
194414.484001\\
238461.251998\\
206435.582000\\
199757.081001\\
213866.972000\\
225913.565002\\
196546.166000\\
213463.575996\\
195959.438003\\
206839.480995\\
220540.803001\\
209264.978004\\
197839.481994\\
392216.760002\\
199549.319000\\
197799.188003\\
203430.643997\\
208214.865997\\
213407.930999\\
202921.930000\\
215385.958999\\
220633.361999\\
197105.193000\\
197583.671005\\
192261.052001\\
194151.289001\\
215815.277999\\
194891.460998\\
201080.104003\\
198913.440002\\
202933.259002\\
198982.377998\\
195121.241996\\
202507.513999\\
197895.453002\\
199605.223999\\
210042.898002\\
197364.318000\\
198122.929000\\
195653.922996\\
210346.429000\\
215784.079002\\
210709.113998\\
288172.521995\\
207316.75\\
204968.500999\\
196863.434997\\
200143.079002\\
217198.010002\\
208987.342994\\
207090.613998\\
194524.899002\\
199193.183006\\
221173.065002\\
195255.782997\\
195518.802001\\

    \hline
  \end{tabular}
  \caption{Tabela de tempo de processamento}
\end{table}
\newpage
\subsection{Médias e gráficos}
Devido ao grande número de dados decidimos por representar graficamente apenas as
médias e desvios de cada opção do \emph{menu}[\ref{itm:menuj}].\\
\subsubsection{Cálculos}
A média calculada entre os pontos foi a média simples, em outras palavras, denominamos a 
média como \boxed{\mu} então, \boxed{\mu = \frac{1}{N}\sum_{i=1}^{N}\left(x_i\right)}, onde N
é o número de pontos, no caso desse projeto temos N=100 medidas para cada uma das opções de menu[\ref{itm:tcpserver}].\\
Já o desvio padrão amostral, denominado \boxed{\sigma} é calculado como \boxed{\sigma = \sqrt{\frac{1}{N}\sum_{i=1}^{N}\left(x_i - \mu\right)^2}}\\
Seja \boxed{\mu_t}, \boxed{\mu_p}, \boxed{\mu_c}, a média do tempo total, média do tempo de processamento e média do tempo de comunicação,
respectivamente, e o desvio padrão do tempo de comunicação, de
nominado \boxed{\sigma_c}.
\newpage
\begin{figure}[!htb]
  \centering
  \subfloat[Gráficos Tempo de comunicação e total]{
  \includegraphics[width=300px,height=500px]{graficos.png}
  \label{graficos}
  }
\end{figure}
\newpage
\begin{table}
  \centering
  \begin{tabular}{|c|c|c|c||c|}
    \hline
    Menu & \boxed{\mu_t} & \boxed{\mu_p} & \boxed{\mu_c} &  \boxed{\pm \sigma_c} \\
    \hline
    1 & 209802.51 & 2313.57 & 207488.94 & 30464.18 (14,68\%)\\
    2 & 205886.77 & 1913.04 & 203973.73  & 12627.63 (6,19\%)\\
    3 & 220287.99 & 2250.85 & 218037.14 & 86018.67 (3,95\%)\\
    4 & 432944.65 & 11673.84 & 421270.81 & 44415.33 (10,54\%)\\
    5 & 411566.99 & 1932.92 & 409634.07 & 28004.75 (6,38\%)\\
    6 & 205504.94 & 1867.76 & 203637.18 & 22187.58 (28,89\%)\\
    \hline
  \end{tabular}
  \caption{Tabela com médias e desvios para RMI}
  \centering
  \begin{tabular}{|c|c|c|c||c|}
    \hline
    Menu & \boxed{\mu_t} & \boxed{\mu_p} & \boxed{\mu_c} &  \boxed{\pm \sigma_c} \\
    \hline
    1 & 202560.05 & 60.41 & 202499.64 & 10252.64 (5,06\%)\\
    2 & 403655.22 & 201822.13 & 201833.09 & 7159.37 (3,54\%)\\
    3 & 404796.77 & 202373.30 & 202423.47 & 4133.69 (2,04\%)\\
    4 & 399065.79 & 563.89 & 398501.90 & 100786.75 (25,29\%)\\
    5 & 855097.01 & 607144.72 & 247952.29 & 51358.97 (20,71\%)\\
    6 & 400975.64 & 200207.89 & 200767.75 & 5832.36 (2,91\%)\\
    \hline
  \end{tabular}
  \caption{Tabela com médias e desvios para protocolo TCP}
\end{table}
\subsection{Discussão}
Vamos definir três componentes no cálculo comparativo entre o TCP, implementado em C, o qual chamaremos de TCP nativo, e o RMI, que utiliza
TCP como protocolo da camada de transporte. São eles: 
\begin{enumerate}[I.]
  \label{sec:componentes}
\item \label{itm:simplicidade} \emph{Simplicidade - [S]}: Relativo ao tempo para escrever o código fonte do projeto, a quantidade de linhas escritas e complexidade de implementação.
\item \label{itm:tempc} \emph{Tempo de comunicacao - [TC]}: Tempo de comunicação, já calculado.
\item \label{itm:tempp} \emph{Tempo de processamento - [TP]}: Tempo de processamento, já calculado.
\end{enumerate}
\\
Para o cálculo comparativo entre os dois sistemas colocamos peso maior nos tempos de comunicação pois, em nosso projeto queremos focar essa variável porém, há casos em que o importante para o projeto seja
uma implementação rápida, de baixo complexidade de organização entre outros fatores que favorecem a simplicidade [S]. O tempo de processamento recebeu o menor peso pois, consideramos que essa componente varia muito entre aplicações e algoritmos utilizados por elas.
Visto que o sistema escrito em \emph{Java}, ou seja, o RMI é muito mais simples do que o escrito em C, o projeto de TCP nativo. Também, há opções do menu na qual o
RMI obtem menor tempo total devido ao tempo de processamento, nota-se que o mesmo foi superior, ou seja, foi na média 279.68\% mais rápido, isso explica-se ao uso de
banco de dados ao invéz de um simples leitura e escrita de arquivos, utilizando árvore AVL\footnote{Vide projeto 1 e 2} como estrutura. Já, para a componente [TC]
(como citado acima a com maior peso no projeto), temos que na média o TCP nativo foi superior em 13.92\% ao RMI, chegando a 65.20\% no caso do menu 5, onde há, 
uma maior troca de mensagens entre o cliente e servidor antes de parar o cronômetro.
\end{enumerate}
\section{Conclusão}
Como escolhemos a componente tempo de comunicação, temos que o sistema TCP nativo, escrito na linguagem C, foi mais rápido do que o RMI, escrito em JAVA. Porém,
é importante notar que a diferença é pequena em comparação ao ganho em termos das componentes de simplicidade e tempo de processamento do RMI. Logo, conclui-se
que dependendo da aplicação que será desenvolvida pode-se escolher o RMI como solução, já que provê uma abstração maior ao programador, todavia se o projeto
requer uma menor abstração e um tempo de comunicação menor, então a escolha fica com o TCP nativo.
\section{Código fonte \textit{(.java)}}
\lstset{ 
   basicstyle=\footnotesize,        % the size of the fonts that are used for the code
   breaklines=true,                 % sets automatic line breaking
   captionpos=b,                    % sets the caption-position to bottom
   frame=single,                    % adds a frame around the code
   language=Java,		    % the language of the code
   numbers=left,                    % where to put the line-numbers; possible values are (none, left, right)
   numbersep=5pt,                   % how far the line-numbers are from the code
   rulecolor=\color{black},         % if not set, the frame-color may be changed on line-breaks within not-black text (e.g. comments (green here))
   showspaces=false,                % show spaces everywhere adding particular underscores; it overrides 'showstringspaces'
   showstringspaces=false,          % underline spaces within strings only
   showtabs=false,                  % show tabs within strings adding particular underscores
   tabsize=2,                       % sets default tabsize to 2 spaces
}
\subsection{Pacote \textit{common}}
\subsubsection{\label \emph{CommonVars.java}}
\lstinputlisting{../src/common/CommonVars.java}
\subsubsection{\label\emph{InOut.java}}
\lstinputlisting{../src/common/InOut.java}
\subsubsection{\label\emph{Tempo.java}}
\lstinputlisting{../src/common/Tempo.java}
\subsection{Pacote \textit{biblioteca}}
\subsubsection{\label\emph{Biblioteca.java}}
\lstinputlisting{../src/biblioteca/Biblioteca.java}
\subsubsection{\label\emph{BibliotecaImpl.java}}
\lstinputlisting{../src/biblioteca/BibliotecaImpl.java}
\subsubsection{\label\emph{Livro.java}}
\lstinputlisting{../src/biblioteca/Livro.java}
\subsubsection{\label\emph{Menu.java}}
\lstinputlisting{../src/biblioteca/Menu.java}
\subsubsection{\label\emph{Usuario.java}}
\lstinputlisting{../src/biblioteca/Usuario.java}
\subsection{Pacote \textit{server}}
\subsubsection{\label\emph{Server.java}}
\lstinputlisting{../src/server/Server.java}
\subsection{Pacote \textit{client}}
\subsubsection{\label\emph{Client.java}}
\lstinputlisting{../src/client/Client.java}
\subsubsection{\label\emph{Tester.java}}
\lstinputlisting{../src/client/Tester.java}
\end{document}
